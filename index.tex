% Options for packages loaded elsewhere
% Options for packages loaded elsewhere
\PassOptionsToPackage{unicode}{hyperref}
\PassOptionsToPackage{hyphens}{url}
\PassOptionsToPackage{dvipsnames,svgnames,x11names}{xcolor}
%
\documentclass[
  12pt,
  letterpaper,
  DIV=11,
  numbers=noendperiod]{scrreprt}
\usepackage{xcolor}
\usepackage{amsmath,amssymb}
\setcounter{secnumdepth}{5}
\usepackage{iftex}
\ifPDFTeX
  \usepackage[T1]{fontenc}
  \usepackage[utf8]{inputenc}
  \usepackage{textcomp} % provide euro and other symbols
\else % if luatex or xetex
  \usepackage{unicode-math} % this also loads fontspec
  \defaultfontfeatures{Scale=MatchLowercase}
  \defaultfontfeatures[\rmfamily]{Ligatures=TeX,Scale=1}
\fi
\usepackage{lmodern}
\ifPDFTeX\else
  % xetex/luatex font selection
\fi
% Use upquote if available, for straight quotes in verbatim environments
\IfFileExists{upquote.sty}{\usepackage{upquote}}{}
\IfFileExists{microtype.sty}{% use microtype if available
  \usepackage[]{microtype}
  \UseMicrotypeSet[protrusion]{basicmath} % disable protrusion for tt fonts
}{}
\makeatletter
\@ifundefined{KOMAClassName}{% if non-KOMA class
  \IfFileExists{parskip.sty}{%
    \usepackage{parskip}
  }{% else
    \setlength{\parindent}{0pt}
    \setlength{\parskip}{6pt plus 2pt minus 1pt}}
}{% if KOMA class
  \KOMAoptions{parskip=half}}
\makeatother
% Make \paragraph and \subparagraph free-standing
\makeatletter
\ifx\paragraph\undefined\else
  \let\oldparagraph\paragraph
  \renewcommand{\paragraph}{
    \@ifstar
      \xxxParagraphStar
      \xxxParagraphNoStar
  }
  \newcommand{\xxxParagraphStar}[1]{\oldparagraph*{#1}\mbox{}}
  \newcommand{\xxxParagraphNoStar}[1]{\oldparagraph{#1}\mbox{}}
\fi
\ifx\subparagraph\undefined\else
  \let\oldsubparagraph\subparagraph
  \renewcommand{\subparagraph}{
    \@ifstar
      \xxxSubParagraphStar
      \xxxSubParagraphNoStar
  }
  \newcommand{\xxxSubParagraphStar}[1]{\oldsubparagraph*{#1}\mbox{}}
  \newcommand{\xxxSubParagraphNoStar}[1]{\oldsubparagraph{#1}\mbox{}}
\fi
\makeatother


\usepackage{longtable,booktabs,array}
\usepackage{calc} % for calculating minipage widths
% Correct order of tables after \paragraph or \subparagraph
\usepackage{etoolbox}
\makeatletter
\patchcmd\longtable{\par}{\if@noskipsec\mbox{}\fi\par}{}{}
\makeatother
% Allow footnotes in longtable head/foot
\IfFileExists{footnotehyper.sty}{\usepackage{footnotehyper}}{\usepackage{footnote}}
\makesavenoteenv{longtable}
\usepackage{graphicx}
\makeatletter
\newsavebox\pandoc@box
\newcommand*\pandocbounded[1]{% scales image to fit in text height/width
  \sbox\pandoc@box{#1}%
  \Gscale@div\@tempa{\textheight}{\dimexpr\ht\pandoc@box+\dp\pandoc@box\relax}%
  \Gscale@div\@tempb{\linewidth}{\wd\pandoc@box}%
  \ifdim\@tempb\p@<\@tempa\p@\let\@tempa\@tempb\fi% select the smaller of both
  \ifdim\@tempa\p@<\p@\scalebox{\@tempa}{\usebox\pandoc@box}%
  \else\usebox{\pandoc@box}%
  \fi%
}
% Set default figure placement to htbp
\def\fps@figure{htbp}
\makeatother


% definitions for citeproc citations
\NewDocumentCommand\citeproctext{}{}
\NewDocumentCommand\citeproc{mm}{%
  \begingroup\def\citeproctext{#2}\cite{#1}\endgroup}
\makeatletter
 % allow citations to break across lines
 \let\@cite@ofmt\@firstofone
 % avoid brackets around text for \cite:
 \def\@biblabel#1{}
 \def\@cite#1#2{{#1\if@tempswa , #2\fi}}
\makeatother
\newlength{\cslhangindent}
\setlength{\cslhangindent}{1.5em}
\newlength{\csllabelwidth}
\setlength{\csllabelwidth}{3em}
\newenvironment{CSLReferences}[2] % #1 hanging-indent, #2 entry-spacing
 {\begin{list}{}{%
  \setlength{\itemindent}{0pt}
  \setlength{\leftmargin}{0pt}
  \setlength{\parsep}{0pt}
  % turn on hanging indent if param 1 is 1
  \ifodd #1
   \setlength{\leftmargin}{\cslhangindent}
   \setlength{\itemindent}{-1\cslhangindent}
  \fi
  % set entry spacing
  \setlength{\itemsep}{#2\baselineskip}}}
 {\end{list}}
\usepackage{calc}
\newcommand{\CSLBlock}[1]{\hfill\break\parbox[t]{\linewidth}{\strut\ignorespaces#1\strut}}
\newcommand{\CSLLeftMargin}[1]{\parbox[t]{\csllabelwidth}{\strut#1\strut}}
\newcommand{\CSLRightInline}[1]{\parbox[t]{\linewidth - \csllabelwidth}{\strut#1\strut}}
\newcommand{\CSLIndent}[1]{\hspace{\cslhangindent}#1}



\setlength{\emergencystretch}{3em} % prevent overfull lines

\providecommand{\tightlist}{%
  \setlength{\itemsep}{0pt}\setlength{\parskip}{0pt}}



 


\usepackage{tikz}
\usetikzlibrary{arrows.meta, positioning, shapes.geometric}
\KOMAoption{captions}{tableheading}
\makeatletter
\@ifpackageloaded{bookmark}{}{\usepackage{bookmark}}
\makeatother
\makeatletter
\@ifpackageloaded{caption}{}{\usepackage{caption}}
\AtBeginDocument{%
\ifdefined\contentsname
  \renewcommand*\contentsname{Table of contents}
\else
  \newcommand\contentsname{Table of contents}
\fi
\ifdefined\listfigurename
  \renewcommand*\listfigurename{List of Figures}
\else
  \newcommand\listfigurename{List of Figures}
\fi
\ifdefined\listtablename
  \renewcommand*\listtablename{List of Tables}
\else
  \newcommand\listtablename{List of Tables}
\fi
\ifdefined\figurename
  \renewcommand*\figurename{Figure}
\else
  \newcommand\figurename{Figure}
\fi
\ifdefined\tablename
  \renewcommand*\tablename{Table}
\else
  \newcommand\tablename{Table}
\fi
}
\@ifpackageloaded{float}{}{\usepackage{float}}
\floatstyle{ruled}
\@ifundefined{c@chapter}{\newfloat{codelisting}{h}{lop}}{\newfloat{codelisting}{h}{lop}[chapter]}
\floatname{codelisting}{Listing}
\newcommand*\listoflistings{\listof{codelisting}{List of Listings}}
\makeatother
\makeatletter
\makeatother
\makeatletter
\@ifpackageloaded{caption}{}{\usepackage{caption}}
\@ifpackageloaded{subcaption}{}{\usepackage{subcaption}}
\makeatother
\usepackage{bookmark}
\IfFileExists{xurl.sty}{\usepackage{xurl}}{} % add URL line breaks if available
\urlstyle{same}
\hypersetup{
  pdftitle={THESIS},
  pdfauthor={BENNOUR MOHAMED HSIN},
  colorlinks=true,
  linkcolor={blue},
  filecolor={Maroon},
  citecolor={Blue},
  urlcolor={Blue},
  pdfcreator={LaTeX via pandoc}}


\title{THESIS}
\author{BENNOUR MOHAMED HSIN}
\date{2025-12-03}
\begin{document}
\maketitle

\renewcommand*\contentsname{Table of contents}
{
\hypersetup{linkcolor=}
\setcounter{tocdepth}{2}
\tableofcontents
}
\listoffigures
\listoftables

\bookmarksetup{startatroot}

\chapter*{Preface}\label{preface}
\addcontentsline{toc}{chapter}{Preface}

\markboth{Preface}{Preface}

This is a Quarto book.

To learn more about Quarto books visit
\url{https://quarto.org/docs/books}.

\bookmarksetup{startatroot}

\chapter{Introduction}\label{introduction}

\section{General background}\label{general-background}

The Covid 19 pandemic showed structural challenges of national economies
all over the world, specifically the fragility of neoliberal policies in
times of crisis and the lack of industrial and economic resilience. Six
years after the fact, our societies are now confronted with inevitable
novel challenges and looming shocks. We are already witnessing the
consequences of AI development as it paves the way for a new
technological revolution that would render most local economies obsolete
and cause massive unemployment in white collar sectors. Sadly, that
doesn't seem to be the end of the shocks the world has seen recently as
the war in Ukraine, Gaza, Iran, and most importantly the current,
chaotic trade wars, seem to foster ever increasing uncertainties. Facing
all this, policy makers are confronted with a simple choice; strategise
and plan for more resilient local economies. In different streams of
literature, resilience is directly related to diversification/variety,
whether in portfolio management in finance, trade partnerships/linkages,
industrial activities, or in terms of knowledge as well. Thus we can
equivalently say that resilience is the capacity to resist and/or adapt
to external shocks by relying on exiting internal capabilities that
evolve in the face of such shocks. This means that for an economy to
survive uncertainties, it needs to evolve, change, and innovate its way
to the other end of the bleak challenges it's confronted with. However,
to evolve, change and innovate, a baseline of knowledge should be
leveraged since the consensus is that variety is a buffer against
external shocks and shields uncertainties. This chain of thoughts, takes
us back to the conceptual basics of sustainable economic development and
growth; the knowledge fabric is what facilitate any long term strategy,
and has been shown in the literature as clear catalyst for societal
prosperity and economic resilience. For this reason the study of
knowledge is detrimental for policy making, and understanding how to
increase its diversity is more relevant than ever before.

Knowledge is a set of information that covers one or many topics, and
its characteristics are contingent on the different forms it can take or
how it was created, generally speaking, academia and businesses are the
main knowledge creators in any economy through research and patents.
Essentially, knowledge can be codified (accessible by anyone through any
medium), or tacit (personal information based on social connections,
intuition, experience, etc, that's hard to share with others). The
consensus in the literature is that the main driver of competitive
advantage for firms is the tacit form of knowledge, which is also widely
acknowledged that it's space dependent. However, the knowledge produced
by firms can be reliably seen in patents, although they capture codified
information, they also reveal tacit knowledge and its geographic
footprint in space. This means that its detrimental to assess the
knowledge in patenting activities (we refer to this knowledge as
technologies), and more so to focus on the local aspect of these
activities i.e: sub-national regions.

This framing, however, is not at all new or novel. In fact, this is the
entire aim of the literature of the geography of innovation; to study
how innovation is created and diffused to different actors in different
geographical contexts. Specifically, relatedness and economic complexity
(REC) is one of the main streams of literature that focus on the
relationships between activities and geographies. The conceptual and
methodological framework that REC provides is widely used and adopted in
academia and among policy practitioners and was one of the main
contributors to the smart specialisation policy literature. The ideas
embedded in this framework, to put it simply, rely on the premise of
spatial dependence of tacit knowledge in local/regional
economies/geographies and focus on simplifying these relationships using
network science to model the relationships between knowledge and
regions. Albeit these simplifications provide valuable scope for
analysis and interpretation, the cost from the loss of granular
information implies that there's much more conceptual, methodological,
and empirical work needed. The reason for this is because the loss of
information bias the empirical interpretation in the sense that we end
up with a homogeneous implication with weak regards to the regional and
national contexts as well as the technological characteristics. This
work is motivated by this gap and aims to simply contextualise the study
of knowledge diversification using the same granular information
publicly available and commonly used in the REC literature. The idea is
simple, account for endogenous and exogenous contexts using granular
data and understand the contexts and contingencies that drives regional
diversification.

\section{Problem statement}\label{problem-statement}

The main problem that this works aims to solve is directly embedded in
the methodological and empirical framework of REC. REC models
diversification through network aggregation based on co-location and
co-occurrence patterns. Using these patterns, different aggregations are
used to quantify relatedness (the frequency of observing a pair of
activities in the same region), and relatedness density (how much of the
activities frequently observed together a region has). However, these
measures are often interpreted not as aggregations of frequent
observations but rather as relationship models. Empirically,
diversification is studied using these constructs as predictors of
entry, that is a region's entry to a new specialisation in a new
technology, often considered as a binary outcome. In here we briefly
outline the big picture in the REC methodological and empirical
framework, its conceptual issues, its empirical consequences, and
highlight the research gap.

\subsection{Relatedness, relatedness density, and
diversification}\label{relatedness-relatedness-density-and-diversification}

Relatedness and relatedness density are essentially measures of
proximity. In a sense they describe how close two technologies are close
to each other, or how close a given technology is to a region given its
portfolio of technologies. To further decompose the problem here, we
will first establish the methodological constructs for proximity
measures. For relatedness that's co-occurrence, and for relatedness
density that's the linear aggregation of relatedness.

First, co-occurrence is essentially the frequency of observing two
activities together. In the REC literature, this frequency describes the
strength of the relationship. Activities frequently observed together
are more related than the pairs rarely observed together.

Second, linear aggregation of relatedness essentially measures the
percentage of co-located technologies in a region that are related to a
reference technology. Thus, we can think about relatedness density as
the link between related technologies and co-located technologies. The
idea in the REC literature states that relative to a given technology,
the more related technologies a region has, the more likely that it can
develop that technology.

These two constructs are used together to predict the probability that a
region will enter a new technology. The REC literature shows that
relatedness density is consistently associated with higher probabilities
in almost all studies. These results among others were one of the major
latent contributors to smart specialisation strategy (S3) policy. Thus
the consensus in the literature was clear: In order to diversify into
new technologies with the highest likelihood of success, regions must
prioritise investment in related technologies.

\subsection{Empirical consequences}\label{empirical-consequences}

The idea of resilience is not a main focus for the REC literature nor it
is ours. However, falling back to this concept allows us to further
assess the empirical consequences of the mainstream interpretation of
relatedness and relatedness density. The idea is that in order to be
resilient to external shocks and subsequent uncertainties
diversification is key. But what kind of diversification is required and
feasible and how to achieve it is the focus here. The REC literature
tells us that the most likely successful diversification strategy is the
one that targets related capacity in the regions, often referred to as
related variety. However, generalising this recommendation is not that
straight forward. Aggregate regional capacities and their national and
broader geographic contexts differ significantly. The initial landscape
of the regional technological portfolio is detrimental here because this
strategy could favour regions with already diverse portfolio but it's
questionable that regions with limited portfolios would equally benefit.
This is aligned with the concept of path dependency, related variety
without context enforces that dependency and locks regions within their
limited capacities. This brings us back to the core focus of in this
work; context is key. However, context on its own here might not be
enough since path dependency of related variety is a direct result of
how relatedness and relatedness density is calculated and interpreted.
The constructs that enable these measures (co-occurrence and linear
aggregation) are the core problem that we're highlighting here. The
reason behind this specific focus relied on the implicit assumptions
embedded in these methodological constructs.

Co-occurrence assumes that technologies frequently observed together are
likely related. Although this is within the boundaries of common sense,
it's highly unlikely that's actually the case. A frequency measure is
only informative when we have more observations than items---that is,
more geographies than technologies. Almost always, we will have more
technologies than geographies, this means that relatedness is at best
noisy. Additionally, relatedness as interpreted in the literature
quantifies the relationship between pairs of technologies. However, as
it stands, there's no differentiation in the direction of that
relationship thus, assuming that the relationship between two
technologies is symmetrical. Albeit this assumption in itself is not
problematic, it exacerbates the linearity issue when we measure
relatedness density as we loose information in co-occurrence, symmetry,
and linear aggregations. This takes us to the final issue we would like
to highlight; relatedness density. Simply put, relatedness density
measures the sum of technologies related to a reference technology
present in a given region. The implicit assumption in here is that
technologies are linked through linear combinations, and those
combinations predict the likelihood of successful diversification.
However, relatedness density is often interpreted as a value that
quantifies the existing requirements a region has relative to a
technology, whereas the sum of existing related technologies do not
inform us on the actual requirements.

In summary, relatedness and relatedness density measures suffer from
diverse methodological issues embedded in the implicit assumptions in
their core constructs. Co-occurrence and linear aggregation of observed
frequency are misinterpreted, accrues information loss, and poorly
handles the granular data often used. This means that the empirical and
methodological work ahead must account for these issues to further
contextualise the study of diversification strategies.

\subsection{Research problem}\label{research-problem}

In the light of all the mentioned in this section, we fall back again on
the core idea that we started this text with; How can we contextualise
diversification strategies? The answer to this question is multi-layered
and complex. In this section we started by outlining the importance of
diversification strategies for regions into new technologies via
patenting activities. We explained that the REC literature provides
interesting methodological and conceptual framework of analysis and
showed that despite their usefulness they suffer from structural issues
that limit the advantages of the used granular data, thus limit the
incorporation of a broader context empirically. Essentially, the
research problem we focus on here, is both methodological and empirical
in nature. We highlighted the structural methodological issues as the
research gap which will be the focus of our methodological and empirical
contribution.

To address the methodological gaps in REC, we propose a multilayered
contextual framework that distinguishes between different contextual
factors. Relatedness measures alone conflate these dimensions. By
decomposing context into endogenous technology-region fit and exogenous
environmental conditions, we can identify which combinations enable
successful diversification across different regional contexts, following
this design:

Endogenous factors (within-region characteristics):

\begin{itemize}
\tightlist
\item
  Technology-specific attributes (network position, centrality,
  embeddedness)
\item
  Regional knowledge infrastructure (coherence between technology
  relationships and regional specialization patterns)
\item
  Regional technological volume (number of patents in a specific
  technology)
\end{itemize}

Exogenous factors (beyond-region influences):

\begin{itemize}
\tightlist
\item
  National entrepreneurial ecosystem characteristics (institutions,
  financing, R\&D culture)
\item
  Spatial spillovers from neighbouring regions (outcome diffusion,
  knowledge flows)
\item
  Geographic proximity to coherent knowledge clusters
\end{itemize}

The interaction between these layers is critical: endogenous capacity
determines what a region can develop, while exogenous context shapes
realisation possibilities. For instance, a technology may align with
regional knowledge infrastructure (endogenous coherence), but without
supportive national institutions or proximity to innovative neighbours
(exogenous support), entry probability remains low.

\includegraphics[width=9.3in,height=19.05in]{intro_files/figure-latex/mermaid-figure-1.png}

\section{Objective}\label{objective}

The objective from this work is to extend the methodological framework
of relatedness starting from the structural issues it suffers from. In
order to show how our methodological contribution benefit the study of
regional technological diversification we rely on an empirical study
that shows how broader context can be included and how such context
inform granular policy insights. The broader context we aim to include
empirically relied on relaxing the implicit relatedness assumptions,
explicitly include the endogenous characteristics of the technologies
and regions contingent on the regional knowledge infrastructure while
simultaneously account for the characteristics of the national
ecosystems. With such an approach we end up accounting for more
contextual layers than the mainstream approach.

\section{Research questions and
hypothesis}\label{research-questions-and-hypothesis}

\textbf{RQ1: How to further contextualise diversification strategy based
on the relatedness framework?}

\begin{itemize}
\tightlist
\item
  H1a: Asymmetric relatedness measures better predict technology entry
  than symmetric measures.
\end{itemize}

\textbf{RQ2: Is successful diversification contingent on regional
knowledge infrastructure?}

\begin{itemize}
\item
  H2a: Technology-specific characteristics impact on diversification is
  contingent on the regional knowledge coherence.
\item
  H2b: Technology-specific characteristics impact on diversification is
  contingent on the regional knowledge stock.
\end{itemize}

\textbf{RQ3: Does the national ecosystem influence diversification?}

\begin{itemize}
\tightlist
\item
  H3a: National Entrepreneurial Ecosystem characteristics positively
  affects regional entry into new technologies.
\end{itemize}

\textbf{RQ4: What role does space have? Do neighbouring regions
influence diversification?}

\begin{itemize}
\tightlist
\item
  H4a: Spatial spillovers of outcome from neighbouring regions
  positively affect technology entry.
\item
  H4b: Neighbouring regions' technological characteristics influence
  focal region diversification outcomes.
\item
  H4c: Geographic proximity to regions with coherent knowledge
  infrastructure increases entry probability.
\end{itemize}

\section{Structure}\label{structure}

We structure this work around our methodological and empirical
contributions. Given the complex nature of the problems we aim to solve,
we will first start with more literature context that underlines
geography of innovation, and more importantly relatedness and economic
complexity literatures. We then outline our approach in the methodology
chapter where we discuss in more details measures of relatedness and
proximity and provide alternative conceptualisation to modeling the
relationships between technologies and regions. In the same chapter we
also detail other elements that will be relevant to the empirical part
specifically knowledge coherence. In the empirical chapter we outline
the research design and the modeling procedures that implements our
methodological contribution. The results chapter outline the results of
our work and its empirical consequences and interpretation, which we
discuss in more detail in the discussion and conclusion of this work in
which we also outline different further extensions and future research
directions.

\section{Relatedness(?)}\label{relatedness}

\bookmarksetup{startatroot}

\chapter{Literature review}\label{literature-review}

\begin{figure}[H]

{\centering \pandocbounded{\includegraphics[keepaspectratio]{pres_rf_s7/get-out-of-my-head-meme.gif}}

}

\caption{\textbf{THIS IS HARD!}}

\end{figure}%

\bookmarksetup{startatroot}

\chapter{The interconnectedness between regions and
technologies}\label{the-interconnectedness-between-regions-and-technologies}

\section{Methodological Motivation}\label{methodological-motivation}

Traditional relatedness frameworks model diversification through
symmetric co-occurrence matrices and linear aggregation of relatedness
density (Hausmann and Hidalgo 2011). However, as established in our
problem statement, these constructs suffer from three structural
limitations: (1) symmetry assumptions that obscure directional
dependencies between technologies, (2) linear aggregation that loses
granular information about technology-specific requirements, and (3)
noise when technologies outnumber regions.

We address these limitations by replacing traditional measures with
machine learning-derived alternatives. Specifically, we use Random
Forest models to generate: (1) \textbf{asymmetric feature importance
networks (FITS)} that capture directional, hierarchical technology
relationships, replacing symmetric relatedness measures; and (2)
\textbf{predicted probabilities (technological potential)} that estimate
region-specific feasibility of technology adoption, replacing linear
relatedness density. This approach enables the contextualization of
diversification strategies by accounting for technology-specific
characteristics, regional knowledge infrastructure, and broader spatial
dynamics---elements that traditional measures cannot adequately capture.

Before detailing our methodology, we briefly situate this approach
within the broader literature. The Principle of Relatedness (PoR)
formalizes the empirical observation that shared input requirements
(knowledge, resources, capabilities) determine diversification
feasibility (Hidalgo et al. 2018; Hidalgo 2021). Economic Complexity
(EC) complements this by quantifying sophistication patterns (Hidalgo
and Hausmann 2009). While these frameworks have proven valuable for
policy (Zaccaria et al. 2018; E and A 2021), deriving granular,
context-specific implications remains challenging (Hidalgo 2023; Li and
Neffke 2024). Recent work on unrelated diversification (Flávio L.
Pinheiro et al. 2022; Boschma et al. 2023), geographic inequalities
(Flavio L. Pinheiro et al. 2025; Hartmann et al. 2017), and emerging
technologies (Lee et al. 2018; Fessina et al. 2024) highlights the need
for methodologies that capture contextual nuances beyond path dependency
identification.

Our approach responds to this need by modeling technology relationships
and regional capabilities in ways that explicitly incorporate
heterogeneity. The following sections establish our notation system,
describe the Random Forest algorithm, and detail how we construct and
interpret technological potential, FITS, and coherence measures.

\section{Notation and Data Structure}\label{notation-and-data-structure}

We establish consistent notation used throughout:

\textbf{Sets:}

\begin{itemize}
\tightlist
\item
  \(\mathcal{T} = \{t : 1 \le t \le N_T\}\): set of technologies
\item
  \(\mathcal{R} = \{r : 1 \le r \le N_R\}\): set of regions\\
\item
  \(\mathcal{Y} = \{y : 1 \le y \le N_Y\}\): set of years
\item
  \(\mathcal{C} = \{c : 1 \le c \le N_C\}\): set of technology
  categories
\item
  \(\mathcal{K} = \{\kappa : 1 \le \kappa \le N_{\kappa}\}\): set of
  countries
\end{itemize}

We use European Patent Office data (1978-2021) classified at the 4-digit
IPC level, yielding 641 distinct technologies across 345 NUTS2 regions
in 34 European countries. IPC classifications provide hierarchical
structure: section (letter, we also refer to this as categories), class
(two digits), subclass (letter). For example, F16H encompasses Section F
(Mechanical engineering), Class 16 (engineering elements for mechanical
power transmission), and Subclass H (gearing systems). We supplement
patent data with Eurostat regional socio-economic indicators detailed in
subsequent sections.

\section{Revealed Technological
Advantage}\label{revealed-technological-advantage}

We quantify regional specialization using the Revealed Comparative
Advantage (Balassa 1965). Although originally designed for trade data,
this metric has been widely adopted in innovation geography literature.
Following P.-A. Balland and Rigby (2017), we refer to it as Revealed
Technological Advantage (RTA) in our patent context. The RTA measures
relative specialization, enabling simultaneous capture of expertise
depth and portfolio diversity while reducing co-occurrence noise
(Fessina et al. 2024). Despite critiques regarding patent-based
applications (P. Balland and Boschma 2019; Diodato et al. 2023), RTA
aligns with our objective of capturing meaningful technology
relationships through machine learning rather than raw co-occurrence.

The RTA for region \(r\) in technology \(t\) during year \(y\) is:

\[
\text{RTA}_{r,t,y}
= \frac{\displaystyle\frac{X_{r,t,y}}{\sum_{t'} X_{r,t',y}}}
       {\displaystyle\frac{\sum_{r'} X_{r',t,y}}{\sum_{r',t'} X_{r',t',y}}}
= \frac{X_{r,t,y}\,\sum_{r',t'}X_{r',t',y}}
       {\bigl(\sum_{t'}X_{r,t',y}\bigr)\,\bigl(\sum_{r'}X_{r',t,y}\bigr)}
\] Where, \(X_{r,t,y}\): patent count for region \(r\) in technology
\(t\) during year \(y\)

For each year \(y\), we construct the RTA matrix \(\mathbf{R}^{(y)}\)
with entries \(\text{RTA}_{r,t,y}\):

\[
\mathbf{R}^{(y)}
= \bigl[\text{RTA}_{r,t,y}\bigr]_{r=1,\dots,N_{R}}^{t=1,\dots,N_{T}}
\]

These yearly matrices form the foundation for all subsequent modeling.
We then binarize specialization for classification tasks:

\[
z_{r,t,y}
\;=\;
\begin{cases}
1 & \text{if }\text{RTA}_{r,t,y}\ge1 \\
0 & \text{otherwise}
\end{cases}
\]

where \(z_{r,t,y} = 1\) indicates region \(r\) has comparative advantage
(specialization) in technology \(t\) at year \(y\).

\section{Random Forest Algorithm}\label{random-forest-algorithm}

Before detailing our applications, we provide comprehensive context on
the Random Forest algorithm, which fuels our entire methodological
approach. We focus on binary classification where \(y_i \in \{0,1\}\).

\subsection{Algorithm Structure}\label{algorithm-structure}

Random Forest constructs an ensemble of \(B\) decision trees through
bootstrap aggregating (bagging) with random feature subsampling. Each
tree \(T_b\) is built on a bootstrap sample \(\mathcal{D}_b^*\) drawn
with replacement from the original dataset.

\subsection{Tree Construction via Recursive
Partitioning}\label{tree-construction-via-recursive-partitioning}

Each tree recursively partitions the feature space through binary
splits. At node \(t\), we randomly select \(m\) features (typically
\(m = \sqrt{p}\)) and evaluate all possible splits within this subset.
For feature \(j\) and threshold \(\tau\), the split creates two child
nodes: \(t_L = \{i : x_{ij} \leq \tau\}\) and
\(t_R = \{i : x_{ij} > \tau\}\).

\subsection{Gini Impurity as Split
Criterion}\label{gini-impurity-as-split-criterion}

Split quality is assessed via Gini impurity:

\[G(t) = 1 - \sum_{k=0}^{1} p_k^2(t) = 2p_0(t)p_1(t)\]

where \(p_k(t) = n_k(t)/n(t)\) represents the proportion of class \(k\)
observations at node \(t\). Gini impurity quantifies node heterogeneity:
\(G=0\) indicates perfect purity (homogeneous class), while \(G=0.5\)
indicates maximum impurity (equal class distribution). The optimal split
maximizes weighted impurity reduction:

\[\Delta G(j, \tau) = G(t) - \left[\frac{n(t_L)}{n(t)} G(t_L) + \frac{n(t_R)}{n(t)} G(t_R)\right]\]

Weighting by relative node size prevents trivial splits that isolate
single observations into pure but uninformative leaves.

\subsection{Termination and Class
Assignment}\label{termination-and-class-assignment}

Recursive splitting continues until predefined stopping criteria: node
purity (\(G=0\)), minimum node size threshold, or maximum tree depth.
Terminal nodes are assigned the majority class of their constituent
observations.

\subsection{Ensemble Prediction}\label{ensemble-prediction}

For prediction, observation \(\mathbf{x}\) traverses all \(B\) trees.
Final classification aggregates individual tree predictions via majority
voting:

\[\hat{y}(\mathbf{x}) = \text{mode}\{\hat{y}_1(\mathbf{x}), \ldots, \hat{y}_B(\mathbf{x})\}\]

Class probabilities are estimated as the proportion of trees predicting
each class:

\[\hat{P}(y=1|\mathbf{x}) = B^{-1}\sum_{b=1}^{B} \mathbb{I}[\hat{y}_b(\mathbf{x})=1]\]

This probability represents empirical vote share across trees. Values
near 1 indicate strong consensus for class 1 (high confidence), while
values near 0.5 reflect uncertainty. Unlike parametric models, these are
data-driven vote proportions rather than model-based probability
estimates.

\subsection{Variance Reduction
Mechanism}\label{variance-reduction-mechanism}

The algorithm's effectiveness stems from variance reduction through
decorrelated predictions. Bootstrap sampling and random feature
selection reduce inter-tree correlation \(\rho\), yielding ensemble
variance:

\[\text{Var}(\bar{y}) = \rho\sigma^2 + \frac{1-\rho}{B}\sigma^2\]

As \(B\) increases and \(\rho\) decreases, ensemble variance diminishes
while maintaining the low bias of flexible tree models.

\subsection{Feature Importance}\label{feature-importance}

Feature importance quantifies each predictor's contribution by
aggregating Gini impurity reductions:

\[I(j) = \frac{1}{B}\sum_{b=1}^{B} \sum_{t \in T_b : v(t)=j} \Delta G(t)\]

where the sum runs over all nodes using feature \(j\) for splitting.
Higher values indicate features consistently creating purer partitions.
Feature importance measures predictive association rather than causal
effect, and suffers from bias toward high-cardinality features and
correlated predictor sets. It ranks predictive relevance but requires
caution in causal interpretation.

\section{Connecting Random Forest to the Relatedness
Framework}\label{connecting-random-forest-to-the-relatedness-framework}

We apply Random Forest in two complementary ways that replace
traditional relatedness constructs:

\textbf{A. Technological Potential} estimates \(p_{r,t,y}\) =
probability of future specialization in technology \(t\) for region
\(r\). This replaces \textbf{relatedness density}---instead of linearly
aggregating symmetric co-occurrence frequencies, we predict
region-specific feasibility using past technology portfolios. This
captures non-linear interactions and regional heterogeneity that linear
measures miss.

\textbf{B. Feature Importance Technology Space (FITS)} extracts
\(I_{t \to t'}\) = directional dependencies between technologies. This
replaces \textbf{symmetric relatedness}---instead of co-occurrence
frequencies, we identify which technologies predict others' future
adoption. Asymmetry reveals hierarchical structures: if
\(I_{t \to t'} \gg I_{t' \to t}\), technology \(t\) is a prerequisite or
``stepping stone'' toward \(t'\).

Together, these measures enable contextualized diversification analysis:
FITS reveals technology-level hierarchies, Potential quantifies
region-specific feasibility, and their combination allows testing how
regional infrastructure, national ecosystems, and spatial factors
moderate technology adoption patterns.

\includegraphics[width=7.97in,height=6.06in]{potential_files/figure-latex/mermaid-figure-1.png}

\section{Technological Potential}\label{technological-potential}

We follow the methodology of Albora et al. (2023), originally developed
for trade data. The innovation lies in modeling each technology
separately rather than constructing a single global model. For every
technology \(i \in \mathcal{T}\), we train a binary classification model
where:

\begin{itemize}
\tightlist
\item
  \textbf{Outcome}: \(z_{r,i,y}\) (whether region \(r\) specializes in
  technology \(i\) at year \(y\))
\item
  \textbf{Features}:
  \(\{\text{RTA}_{r,t,y-\delta} : t \in \mathcal{T}, t \neq i\}\) (RTA
  values for all other technologies \(\delta\) years prior)
\end{itemize}

This technology-specific approach enables each model to learn unique
dependency patterns. Setting \(\delta = 4\) years balances predictive
performance with data availability such that current capacity predict
future ones (Andreoni and Chang 2019).

\subsection{Training Procedure}\label{training-procedure}

We predict specialization for target years
\(y_t \in \{2008, \ldots, 2018\}\) using data from 1978 onward. For each
target year \(y_t\) and technology \(i\):

\textbf{Training set:} \[
X_{\text{train}} = \{ \text{RTA}_{r,t,y} \mid y \in [1978, y_t - 2\delta], t \neq i\}
\] \[
Y_{\text{train}} = \{z_{r,i,y} \mid y \in [1978 + \delta,  y_t - \delta]\}
\]

\textbf{Test set:} \[
X_{\text{test}} = \{ \text{RTA}_{r,t,y_t-\delta} \mid t \neq i\}
\] \[
Y_{\text{test}} = \{z_{r,i,y_t}\}
\]

This produces 7,051 models (641 technologies × 11 years). Given
computational constraints, we performed cross-validation on a random
sample of four technologies (G06G, B67B, D02J, C08J) and applied the
most frequent optimal parameters across all models:
\texttt{mtry\ =\ 139}, \texttt{trees\ =\ 100}, \texttt{min\_n\ =\ 38}.
Training was conducted in R using the Ranger package (Wright and Ziegler
2017), orchestrated via the targets pipeline (Landau 2021) within the
tidymodels framework.

\subsection{Output: Regional Technological
Potential}\label{output-regional-technological-potential}

Each model produces probabilities
\(p_{r,i,y} = P(z_{r,i,y} = 1 \mid \text{RTA}_{r,\cdot,y-\delta})\)
representing the likelihood that region \(r\) develops specialization in
technology \(i\) at year \(y\) given its past portfolio. These
probabilities constitute \textbf{regional technological potential}---a
forward-looking, region-specific measure of diversification feasibility.

Aggregating across technologies yields average regional potential:

\[p_{r,y} = \frac{\sum_{t} p_{r,t,y}}{N_T}\]

\subsection{Conceptual Interpretation}\label{conceptual-interpretation}

Technological potential differs fundamentally from relatedness density.
While relatedness density assumes technologies combine linearly and
symmetrically, potential:

\begin{enumerate}
\def\labelenumi{\arabic{enumi}.}
\tightlist
\item
  Captures \textbf{non-linear interactions} between technologies through
  Random Forest's decision tree structure
\item
  Allows \textbf{asymmetric dependencies} where technology \(t\) may
  enable \(t'\) but not vice versa
\item
  Produces \textbf{region-specific estimates} rather than universal
  technology-pair relationships
\item
  Reflects \textbf{time-varying dynamics} by training on expanding
  windows of historical data
\end{enumerate}

\subsection{Empirical Implications}\label{empirical-implications}

High potential (\(p_{r,t,y} \approx 1\)) indicates a region's existing
portfolio strongly predicts future specialization in technology
\(t\)---the region likely possesses necessary complementary
capabilities. Low potential (\(p_{r,t,y} \approx 0\)) suggests
capability gaps despite potential relatedness. Crucially, potential
varies across regions for the same technology, enabling analysis of how
regional knowledge infrastructure, national ecosystems, and spatial
factors moderate diversification feasibility---our core research
questions.

\section{Feature Importance Technology Space
(FITS)}\label{feature-importance-technology-space-fits}

Traditional technology networks use patent citations or co-occurrence
patterns. Citations suffer from three limitations: (1) examiner-added
citations may not reflect actual knowledge flows, (2) aggregating
patent-level citations to technology-level relationships obscures
directionality, and (3) citation networks are backward-looking rather
than predictive (Fessina et al. 2024). Co-occurrence networks face the
issues outlined in our problem statement: symmetry, noise, and linear
assumptions.

FITS addresses these limitations by constructing asymmetric, predictive
networks from machine learning. Rather than inferring relationships from
co-occurrence, FITS extracts directional dependencies from the feature
importance scores of our Technological Potential models.

\subsection{FITS Construction}\label{fits-construction}

Recall that for each technology \(i\), we trained a Random Forest model
predicting \(z_{r,i,y}\) using
\(\{\text{RTA}_{r,t,y-\delta} : t \neq i\}\) as features. The feature
importance \(I_i(t)\) quantifies how much technology \(t\) contributes
to predicting future specialization in technology \(i\) across all
regions and time periods in the training data.

We formalize FITS as a directed, weighted network \(G = (V, E, W)\)
where:

\begin{itemize}
\tightlist
\item
  \textbf{Nodes} \(V = \mathcal{T}\): the set of technologies
\item
  \textbf{Edges} \(E\): directed connections \((t \to i)\) for all
  \(t, i \in \mathcal{T}, t \neq i\)
\item
  \textbf{Weights} \(W_{t \to i} = I_i(t)\): feature importance of
  technology \(t\) in model predicting technology \(i\)
\end{itemize}

We normalize weights within each target technology's model:

\[
W_{t \to i} = \frac{I_i(t)}{\sum_{t' \neq i} I_i(t')}
\]

ensuring that for each technology \(i\), incoming edge weights sum to 1:
\(\sum_{t \neq i} W_{t \to i} = 1\).

\subsection{Mathematical Formulation}\label{mathematical-formulation}

From the Random Forest algorithm, feature importance for technology
\(t\) in model \(M_i\) (predicting technology \(i\)) is:

\[
I_i(t) = \frac{1}{B}\sum_{b=1}^{B} \sum_{n \in T_b : v(n)=t} \Delta G(n)
\]

where the sum runs over all nodes \(n\) in all trees \(T_b\) that split
on feature \(t\), and \(\Delta G(n)\) is the Gini impurity reduction at
node \(n\). Technologies that consistently create purer partitions when
predicting \(i\) receive higher importance scores.

The full FITS network aggregates these relationships across all 641
technologies, producing a \(641 \times 641\) weighted adjacency matrix
(excluding self-loops).

\subsection{Asymmetry and Hierarchy}\label{asymmetry-and-hierarchy}

Crucially, FITS is \textbf{asymmetric}: \(W_{t \to i} \neq W_{i \to t}\)
in general. This asymmetry captures hierarchical technological
dependencies:

\begin{itemize}
\tightlist
\item
  If \(W_{t \to i} \gg W_{i \to t}\), technology \(t\) is a
  \textbf{prerequisite} or \textbf{stepping stone} toward \(i\)
  (expertise in \(t\) predicts future \(i\), but not vice versa)
\item
  If \(W_{t \to i} \approx W_{i \to t}\), technologies are
  \textbf{complementary peers} (mutual predictive relationships)
\item
  If \(W_{t \to i} \ll W_{i \to t}\), technology \(i\) is a prerequisite
  toward \(t\)
\end{itemize}

This hierarchical structure is invisible to symmetric co-occurrence
measures and reveals technological trajectories: regions can identify
which current capabilities enable paths toward desired future
technologies.

\subsection{Conceptual
Interpretation}\label{conceptual-interpretation-1}

FITS edges represent \textbf{predictive dependencies} based on
historical diversification patterns across all European regions. A
strong edge \(t \to i\) indicates that regions with RTA in technology
\(t\) at time \(y-\delta\) frequently developed RTA in technology \(i\)
by time \(y\), even after controlling for all other technologies. This
is fundamentally different from co-occurrence (which measures
simultaneous presence) or citations (which measure backward-looking
knowledge flows).

\subsection{Empirical Implications}\label{empirical-implications-1}

FITS enables novel analyses of technological landscapes:

\begin{enumerate}
\def\labelenumi{\arabic{enumi}.}
\tightlist
\item
  \textbf{Path identification}: For a target technology \(i\), examine
  incoming edge weights to identify strong predictors (prerequisites)
\item
  \textbf{Branching points}: High out-degree nodes represent
  foundational technologies enabling diverse future specializations
\item
  \textbf{Category structure}: Aggregating edges by IPC categories
  reveals cross-domain dependencies (e.g., mechanical engineering →
  electronics)
\item
  \textbf{Regional positioning}: Comparing a region's current portfolio
  against FITS edge patterns reveals strategic opportunities and gaps
\end{enumerate}

In our empirical analysis, FITS allows testing whether
technology-specific characteristics (position in the network hierarchy,
in/out-degree patterns, category embeddings) moderate diversification
outcomes---aspects traditional relatedness measures cannot capture.

\subsection{Network Visualization}\label{network-visualization}

\begin{figure}[H]

{\centering \pandocbounded{\includegraphics[keepaspectratio]{resources/technology_network.png}}

}

\caption{Technology Network (based on 2018 patent data)}

\end{figure}%

The figure displays the FITS network with nodes colored by IPC category,
sized by total degree, and transparency reflecting eigenvector
centrality. Labels appear for high-centrality technologies. The
structure reveals dense within-category connections and sparse but
critical between-category bridges.

\section{Coherence: Bridging Technology Networks and Regional
Portfolios}\label{coherence-bridging-technology-networks-and-regional-portfolios}

FITS identifies technology-to-technology relationships. Technological
Potential quantifies regional diversification feasibility.
\textbf{Coherence} bridges these levels by measuring the alignment
between a technology's position in the FITS network and a region's
existing specialization structure.

\subsection{Conceptual Motivation}\label{conceptual-motivation}

Consider two regions, both lacking specialization in technology \(i\),
both having similar potential \(p_{r,i,y}\). However, Region A
specializes in technologies that are strong predictors of \(i\) (high
incoming FITS edges), while Region B specializes in technologies
unrelated to \(i\) in the network. Coherence captures this difference:
Region A has high coherence with \(i\) (its portfolio aligns with
\(i\)'s network prerequisites), while Region B has low coherence
(misalignment).

This metric operationalizes the ``knowledge coherence'' and ``cognitive
proximity'' concepts from innovation literature (Neffke, Henning, and
Boschma 2011; Boschma 2015) using our directional network structure. It
enables testing whether diversification success depends not just on
potential (predicted feasibility) but on the structural fit between
regional portfolios and technology network positions.

\subsection{Mathematical Construction}\label{mathematical-construction}

For each region \(r\), technology \(i\), and IPC category \(c\), we
construct two embedding vectors capturing \(i\)'s directional network
position and compare them to \(r\)'s average embeddings for technologies
in category \(c\).

\textbf{Technology embeddings} (individual technology \(i\)):

\begin{itemize}
\tightlist
\item
  Incoming:
  \(\text{embcat\_to}_{i,c} = \frac{\sum_{t \in c} W_{t \to i}}{|\{t \in c : W_{t \to i} > 0\}|}\)
  (average FITS weight from category \(c\) to technology \(i\))
\item
  Outgoing:
  \(\text{embcat\_from}_{i,c} = \frac{\sum_{t' \in c} W_{i \to t'}}{|\{t' \in c : W_{i \to t'} > 0\}|}\)
  (average FITS weight from technology \(i\) to category \(c\))
\end{itemize}

\textbf{Regional average embeddings} (region \(r\), category \(c\)):

\begin{itemize}
\tightlist
\item
  Incoming:
  \(\overline{\text{embcat\_to}}_{r,c} = \frac{1}{|S_{r,c}|}\sum_{t \in S_{r,c}} \text{embcat\_to}_{t,c}\)
  where \(S_{r,c} = \{t \in c : \text{RTA}_{r,t,y} \geq 1\}\)
\item
  Outgoing:
  \(\overline{\text{embcat\_from}}_{r,c} = \frac{1}{|S_{r,c}|}\sum_{t \in S_{r,c}} \text{embcat\_from}_{t,c}\)
\end{itemize}

\textbf{Coherence} is the cosine similarity between technology \(i\)'s
directional embeddings and region \(r\)'s average directional embeddings
for category \(c\):

\[
\text{Coherence}_{r,i,c,y} = \frac{\mathbf{v}_1 \cdot \mathbf{v}_2}{||\mathbf{v}_1|| \cdot ||\mathbf{v}_2||}
\]

where:

\begin{itemize}
\tightlist
\item
  \(\mathbf{v}_1 = [\text{embcat\_to}_{i,c}, \overline{\text{embcat\_to}}_{r,c}]\)
\item
  \(\mathbf{v}_2 = [\text{embcat\_from}_{i,c}, \overline{\text{embcat\_from}}_{r,c}]\)
\end{itemize}

\includegraphics[width=25.13in,height=5.33in]{potential_files/figure-latex/mermaid-figure-2.png}

\subsection{Interpretation}\label{interpretation}

Coherence ranges from -1 to 1:

\begin{itemize}
\tightlist
\item
  \textbf{High coherence} (\(\approx 1\)): Technology \(i\)'s FITS
  network position (both incoming and outgoing connections to category
  \(c\)) closely matches the average network position of technologies in
  which region \(r\) specializes within category \(c\). The region's
  existing capabilities align with the structural prerequisites and
  consequences of technology \(i\).
\item
  \textbf{Neutral coherence} (\(\approx 0\)): Misalignment between
  technology \(i\)'s relational structure and regional specialization
  patterns.
\item
  \textbf{Negative coherence} (\(\approx -1\)): Technology \(i\)'s
  network position is opposite to the region's specialization structure
  (e.g., \(i\) receives inputs from categories where the region sends
  outputs).
\end{itemize}

\subsection{Empirical Application}\label{empirical-application}

Coherence serves two roles in our empirical analysis:

\begin{enumerate}
\def\labelenumi{\arabic{enumi}.}
\item
  \textbf{Interaction with Potential}: Test whether high potential
  translates to actual diversification only when coherence is also high
  (H2a: technology-specific characteristics moderated by regional
  knowledge coherence)
\item
  \textbf{Regional Infrastructure Measure}: Aggregate coherence across a
  region's non-specialized technologies indicates how well the regional
  portfolio is ``positioned'' in the FITS network for future
  diversification (captures knowledge infrastructure quality)
\end{enumerate}

By incorporating coherence, we test whether successful diversification
requires not just predicted feasibility (potential) and related
capabilities (traditional relatedness), but also structural alignment
between regional portfolios and network prerequisites---a form of
contextualization that traditional measures cannot capture.

\bookmarksetup{startatroot}

\chapter{Summary of Methodological
Framework}\label{summary-of-methodological-framework}

Our approach replaces traditional relatedness constructs with machine
learning-derived measures that enable contextualized diversification
analysis:

\begin{itemize}
\tightlist
\item
  \textbf{Technological Potential} (\(p_{r,t,y}\)): Region-specific,
  non-linear, time-varying probabilities replace linear relatedness
  density
\item
  \textbf{FITS Network} (\(W_{t \to t'}\)): Asymmetric, predictive
  dependencies replace symmetric co-occurrence-based relatedness\\
\item
  \textbf{Coherence} (\(\text{Coherence}_{r,t,c,y}\)): Structural
  alignment between regional portfolios and technology network positions
  captures knowledge infrastructure quality
\end{itemize}

Together, these measures allow testing how diversification is contingent
on regional knowledge infrastructure (RQ2), national ecosystem
characteristics (RQ3), and spatial factors (RQ4) in ways that
traditional relatedness frameworks cannot---addressing the core problem
of contextualizing diversification strategies beyond path dependency
identification.

\bookmarksetup{startatroot}

\chapter{Results}\label{results}

\bookmarksetup{startatroot}

\chapter{Summary}\label{summary}

In summary, this book has no content whatsoever.

\bookmarksetup{startatroot}

\chapter*{References}\label{references}
\addcontentsline{toc}{chapter}{References}

\markboth{References}{References}

\phantomsection\label{refs}
\begin{CSLReferences}{1}{0}
\bibitem[\citeproctext]{ref-albora2023product}
Albora, Giambattista, Luciano Pietronero, Andrea Tacchella, and Andrea
Zaccaria. 2023. {``Product Progression: A Machine Learning Approach to
Forecasting Industrial Upgrading.''} \emph{Scientific Reports} 13 (1):
1481.

\bibitem[\citeproctext]{ref-andreoni2019political}
Andreoni, Antonio, and Ha-Joon Chang. 2019. {``The Political Economy of
Industrial Policy: Structural Interdependencies, Policy Alignment and
Conflict Management.''} \emph{Structural Change and Economic Dynamics}
48: 136--50.

\bibitem[\citeproctext]{ref-balassa1965rca}
Balassa, Bela. 1965. {``Trade Liberalisation and {`Revealed'}
Comparative Advantage 1.''} \emph{The Manchester School} 33 (2):
99--123.

\bibitem[\citeproctext]{ref-balland2019beyond}
Balland, PA, and R Boschma. 2019. {``Smart Specialization: Beyond
Patents.''} \emph{European Commission: Brussels, Belgium}.

\bibitem[\citeproctext]{ref-balland2017geography}
Balland, Pierre-Alexandre, and David Rigby. 2017. {``The Geography of
Complex Knowledge.''} \emph{Economic Geography} 93 (1): 1--23.

\bibitem[\citeproctext]{ref-boschma2015towards}
Boschma, Ron. 2015. {``Towards an Evolutionary Perspective on Regional
Resilience.''} \emph{Regional Studies} 49 (5): 733--51.

\bibitem[\citeproctext]{ref-boschma2023role}
Boschma, Ron, Ernest Miguelez, Rosina Moreno, and Diego B
Ocampo-Corrales. 2023. {``The Role of Relatedness and Unrelatedness for
the Geography of Technological Breakthroughs in Europe.''}
\emph{Economic Geography} 99 (2): 117--39.

\bibitem[\citeproctext]{ref-diodato2023economic}
Diodato, Dario, Lorenzo Napolitano, Emanuele Pugliese, and Andrea
Tacchella. 2023. {``Economic Complexity for Regional Industrial
Strategies.''} Joint Research Centre.

\bibitem[\citeproctext]{ref-albora2021economic}
E, Pugliese, and Tacchella A. 2021. {``Economic Complexity Analytics:
Country Factsheets.''} \url{https://doi.org/10.2760/368138}.

\bibitem[\citeproctext]{ref-fessina2024identifying}
Fessina, Massimiliano, Giambattista Albora, Andrea Tacchella, and Andrea
Zaccaria. 2024. {``Identifying Key Products to Trigger New Exports: An
Explainable Machine Learning Approach.''} \emph{Journal of Physics:
Complexity} 5 (2): 025003.

\bibitem[\citeproctext]{ref-hartmann2017linking}
Hartmann, Dominik, Miguel R Guevara, Cristian Jara-Figueroa, Manuel
Aristarán, and César A Hidalgo. 2017. {``Linking Economic Complexity,
Institutions, and Income Inequality.''} \emph{World Development} 93:
75--93.

\bibitem[\citeproctext]{ref-hausmann2011network}
Hausmann, Ricardo, and César A Hidalgo. 2011. {``The Network Structure
of Economic Output.''} \emph{Journal of Economic Growth} 16: 309--42.

\bibitem[\citeproctext]{ref-hidalgo2021economic}
Hidalgo, César A. 2021. {``Economic Complexity Theory and
Applications.''} \emph{Nature Reviews Physics} 3 (2): 92--113.

\bibitem[\citeproctext]{ref-hidalgo2023policy}
---------. 2023. {``The Policy Implications of Economic Complexity.''}
\emph{Research Policy} 52 (9): 104863.

\bibitem[\citeproctext]{ref-hidalgo2018principle}
Hidalgo, César A, Pierre-Alexandre Balland, Ron Boschma, Mercedes
Delgado, Maryann Feldman, Koen Frenken, Edward Glaeser, et al. 2018.
{``The Principle of Relatedness.''} In \emph{Unifying Themes in Complex
Systems IX: Proceedings of the Ninth International Conference on Complex
Systems 9}, 451--57. Springer.

\bibitem[\citeproctext]{ref-hidalgo2009building}
Hidalgo, César A, and Ricardo Hausmann. 2009. {``The Building Blocks of
Economic Complexity.''} \emph{Proceedings of the National Academy of
Sciences} 106 (26): 10570--75.

\bibitem[\citeproctext]{ref-landau2021targets}
Landau, William Michael. 2021. {``The Targets r Package: A Dynamic
Make-Like Function-Oriented Pipeline Toolkit for Reproducibility and
High-Performance Computing.''} \emph{Journal of Open Source Software} 6
(57): 2959.

\bibitem[\citeproctext]{ref-lee2018early}
Lee, Changyong, Ohjin Kwon, Myeongjung Kim, and Daeil Kwon. 2018.
{``Early Identification of Emerging Technologies: A Machine Learning
Approach Using Multiple Patent Indicators.''} \emph{Technological
Forecasting and Social Change} 127: 291--303.

\bibitem[\citeproctext]{ref-li2024evaluating}
Li, Yang, and Frank MH Neffke. 2024. {``Evaluating the Principle of
Relatedness: Estimation, Drivers and Implications for Policy.''}
\emph{Research Policy} 53 (3): 104952.

\bibitem[\citeproctext]{ref-neffke2011how}
Neffke, Frank, Martin Henning, and Ron Boschma. 2011. {``How Do Regions
Diversify over Time? Industry Relatedness and the Development of New
Growth Paths in Regions.''} \emph{Economic Geography} 87 (3): 237--65.

\bibitem[\citeproctext]{ref-pinheiro2025dark}
Pinheiro, Flavio L, Pierre-Alexandre Balland, Ron Boschma, and Dominik
Hartmann. 2025. {``The Dark Side of the Geography of Innovation:
Relatedness, Complexity and Regional Inequality in Europe.''}
\emph{Regional Studies} 59 (1): 2106362.

\bibitem[\citeproctext]{ref-pinheiro2022time}
Pinheiro, Flávio L, Dominik Hartmann, Ron Boschma, and César A Hidalgo.
2022. {``The Time and Frequency of Unrelated Diversification.''}
\emph{Research Policy} 51 (8): 104323.

\bibitem[\citeproctext]{ref-wright2017ranger}
Wright, Marvin N, and Andreas Ziegler. 2017. {``Ranger: A Fast
Implementation of Random Forests for High Dimensional Data in c++ and
r.''} \emph{Journal of Statistical Software} 77: 1--17.

\bibitem[\citeproctext]{ref-zaccaria2018integrating}
Zaccaria, Andrea, Saurabh Mishra, Masud Z Cader, and Luciano Pietronero.
2018. {``Integrating Services in the Economic Fitness Approach.''}
\emph{World Bank Policy Research Working Paper}, no. 8485.

\end{CSLReferences}




\end{document}
