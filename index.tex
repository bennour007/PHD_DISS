% Options for packages loaded elsewhere
% Options for packages loaded elsewhere
\PassOptionsToPackage{unicode}{hyperref}
\PassOptionsToPackage{hyphens}{url}
\PassOptionsToPackage{dvipsnames,svgnames,x11names}{xcolor}
%
\documentclass[
  12pt,
  letterpaper,
  DIV=11,
  numbers=noendperiod]{scrreprt}
\usepackage{xcolor}
\usepackage[top=2.5cm,bottom=2.5cm,left=3cm,right=2.5cm]{geometry}
\usepackage{amsmath,amssymb}
\setcounter{secnumdepth}{5}
\usepackage{iftex}
\ifPDFTeX
  \usepackage[T1]{fontenc}
  \usepackage[utf8]{inputenc}
  \usepackage{textcomp} % provide euro and other symbols
\else % if luatex or xetex
  \usepackage{unicode-math} % this also loads fontspec
  \defaultfontfeatures{Scale=MatchLowercase}
  \defaultfontfeatures[\rmfamily]{Ligatures=TeX,Scale=1}
\fi
\usepackage{lmodern}
\ifPDFTeX\else
  % xetex/luatex font selection
  \setmainfont[]{Times New Roman}
\fi
% Use upquote if available, for straight quotes in verbatim environments
\IfFileExists{upquote.sty}{\usepackage{upquote}}{}
\IfFileExists{microtype.sty}{% use microtype if available
  \usepackage[]{microtype}
  \UseMicrotypeSet[protrusion]{basicmath} % disable protrusion for tt fonts
}{}
\usepackage{setspace}
\makeatletter
\@ifundefined{KOMAClassName}{% if non-KOMA class
  \IfFileExists{parskip.sty}{%
    \usepackage{parskip}
  }{% else
    \setlength{\parindent}{0pt}
    \setlength{\parskip}{6pt plus 2pt minus 1pt}}
}{% if KOMA class
  \KOMAoptions{parskip=half}}
\makeatother
% Make \paragraph and \subparagraph free-standing
\makeatletter
\ifx\paragraph\undefined\else
  \let\oldparagraph\paragraph
  \renewcommand{\paragraph}{
    \@ifstar
      \xxxParagraphStar
      \xxxParagraphNoStar
  }
  \newcommand{\xxxParagraphStar}[1]{\oldparagraph*{#1}\mbox{}}
  \newcommand{\xxxParagraphNoStar}[1]{\oldparagraph{#1}\mbox{}}
\fi
\ifx\subparagraph\undefined\else
  \let\oldsubparagraph\subparagraph
  \renewcommand{\subparagraph}{
    \@ifstar
      \xxxSubParagraphStar
      \xxxSubParagraphNoStar
  }
  \newcommand{\xxxSubParagraphStar}[1]{\oldsubparagraph*{#1}\mbox{}}
  \newcommand{\xxxSubParagraphNoStar}[1]{\oldsubparagraph{#1}\mbox{}}
\fi
\makeatother


\usepackage{longtable,booktabs,array}
\usepackage{calc} % for calculating minipage widths
% Correct order of tables after \paragraph or \subparagraph
\usepackage{etoolbox}
\makeatletter
\patchcmd\longtable{\par}{\if@noskipsec\mbox{}\fi\par}{}{}
\makeatother
% Allow footnotes in longtable head/foot
\IfFileExists{footnotehyper.sty}{\usepackage{footnotehyper}}{\usepackage{footnote}}
\makesavenoteenv{longtable}
\usepackage{graphicx}
\makeatletter
\newsavebox\pandoc@box
\newcommand*\pandocbounded[1]{% scales image to fit in text height/width
  \sbox\pandoc@box{#1}%
  \Gscale@div\@tempa{\textheight}{\dimexpr\ht\pandoc@box+\dp\pandoc@box\relax}%
  \Gscale@div\@tempb{\linewidth}{\wd\pandoc@box}%
  \ifdim\@tempb\p@<\@tempa\p@\let\@tempa\@tempb\fi% select the smaller of both
  \ifdim\@tempa\p@<\p@\scalebox{\@tempa}{\usebox\pandoc@box}%
  \else\usebox{\pandoc@box}%
  \fi%
}
% Set default figure placement to htbp
\def\fps@figure{htbp}
\makeatother


% definitions for citeproc citations
\NewDocumentCommand\citeproctext{}{}
\NewDocumentCommand\citeproc{mm}{%
  \begingroup\def\citeproctext{#2}\cite{#1}\endgroup}
\makeatletter
 % allow citations to break across lines
 \let\@cite@ofmt\@firstofone
 % avoid brackets around text for \cite:
 \def\@biblabel#1{}
 \def\@cite#1#2{{#1\if@tempswa , #2\fi}}
\makeatother
\newlength{\cslhangindent}
\setlength{\cslhangindent}{1.5em}
\newlength{\csllabelwidth}
\setlength{\csllabelwidth}{3em}
\newenvironment{CSLReferences}[2] % #1 hanging-indent, #2 entry-spacing
 {\begin{list}{}{%
  \setlength{\itemindent}{0pt}
  \setlength{\leftmargin}{0pt}
  \setlength{\parsep}{0pt}
  % turn on hanging indent if param 1 is 1
  \ifodd #1
   \setlength{\leftmargin}{\cslhangindent}
   \setlength{\itemindent}{-1\cslhangindent}
  \fi
  % set entry spacing
  \setlength{\itemsep}{#2\baselineskip}}}
 {\end{list}}
\usepackage{calc}
\newcommand{\CSLBlock}[1]{\hfill\break\parbox[t]{\linewidth}{\strut\ignorespaces#1\strut}}
\newcommand{\CSLLeftMargin}[1]{\parbox[t]{\csllabelwidth}{\strut#1\strut}}
\newcommand{\CSLRightInline}[1]{\parbox[t]{\linewidth - \csllabelwidth}{\strut#1\strut}}
\newcommand{\CSLIndent}[1]{\hspace{\cslhangindent}#1}



\setlength{\emergencystretch}{3em} % prevent overfull lines

\providecommand{\tightlist}{%
  \setlength{\itemsep}{0pt}\setlength{\parskip}{0pt}}



 


\usepackage{tikz}
\usetikzlibrary{arrows.meta, positioning, shapes.geometric}
\usepackage{fontspec}
\setmainfont{Times New Roman}
\KOMAoption{captions}{tableheading}
\makeatletter
\@ifpackageloaded{bookmark}{}{\usepackage{bookmark}}
\makeatother
\makeatletter
\@ifpackageloaded{caption}{}{\usepackage{caption}}
\AtBeginDocument{%
\ifdefined\contentsname
  \renewcommand*\contentsname{Table of contents}
\else
  \newcommand\contentsname{Table of contents}
\fi
\ifdefined\listfigurename
  \renewcommand*\listfigurename{List of Figures}
\else
  \newcommand\listfigurename{List of Figures}
\fi
\ifdefined\listtablename
  \renewcommand*\listtablename{List of Tables}
\else
  \newcommand\listtablename{List of Tables}
\fi
\ifdefined\figurename
  \renewcommand*\figurename{Figure}
\else
  \newcommand\figurename{Figure}
\fi
\ifdefined\tablename
  \renewcommand*\tablename{Table}
\else
  \newcommand\tablename{Table}
\fi
}
\@ifpackageloaded{float}{}{\usepackage{float}}
\floatstyle{ruled}
\@ifundefined{c@chapter}{\newfloat{codelisting}{h}{lop}}{\newfloat{codelisting}{h}{lop}[chapter]}
\floatname{codelisting}{Listing}
\newcommand*\listoflistings{\listof{codelisting}{List of Listings}}
\makeatother
\makeatletter
\makeatother
\makeatletter
\@ifpackageloaded{caption}{}{\usepackage{caption}}
\@ifpackageloaded{subcaption}{}{\usepackage{subcaption}}
\makeatother
\usepackage{bookmark}
\IfFileExists{xurl.sty}{\usepackage{xurl}}{} % add URL line breaks if available
\urlstyle{same}
\hypersetup{
  pdftitle={THESIS},
  pdfauthor={BENNOUR MOHAMED HSIN},
  colorlinks=true,
  linkcolor={blue},
  filecolor={Maroon},
  citecolor={Blue},
  urlcolor={Blue},
  pdfcreator={LaTeX via pandoc}}


\title{THESIS}
\author{BENNOUR MOHAMED HSIN}
\date{2025-12-03}
\begin{document}
\maketitle

\renewcommand*\contentsname{Table of contents}
{
\hypersetup{linkcolor=}
\setcounter{tocdepth}{2}
\tableofcontents
}
\listoffigures
\listoftables

\setstretch{1.5}
\bookmarksetup{startatroot}

\chapter*{Preface}\label{preface}
\addcontentsline{toc}{chapter}{Preface}

\markboth{Preface}{Preface}

This is a Quarto book.

To learn more about Quarto books visit
\url{https://quarto.org/docs/books}.

\bookmarksetup{startatroot}

\chapter{Introduction}\label{introduction}

\section{General background}\label{general-background}

The Covid 19 pandemic showed structural challenges of national economies
all over the world, specifically the fragility of neoliberal policies in
times of crisis and the lack of industrial and economic resilience. Six
years after the fact, our societies are now confronted with inevitable
novel challenges and looming shocks. We are already witnessing the
consequences of AI development as it paves the way for a new
technological revolution that would render most local economies obsolete
and cause massive unemployment in white collar sectors. Sadly, that
doesn't seem to be the end of the shocks the world has seen recently as
the war in Ukraine, Gaza, Iran, and most importantly the current,
chaotic trade wars, seem to foster ever increasing uncertainties. Facing
all this, policy makers are confronted with a simple choice; strategise
and plan for more resilient local economies. In different streams of
literature, resilience is directly related to diversification/variety,
whether in portfolio management in finance, trade partnerships/linkages,
industrial activities, or in terms of knowledge as well. Thus we can
equivalently say that resilience is the capacity to resist and/or adapt
to external shocks by relying on exiting internal capabilities that
evolve in the face of such shocks. This means that for an economy to
survive uncertainties, it needs to evolve, change, and innovate its way
to the other end of the bleak challenges it's confronted with. However,
to evolve, change and innovate, a baseline of knowledge should be
leveraged since the consensus is that variety is a buffer against
external shocks and shields uncertainties. This chain of thoughts, takes
us back to the conceptual basics of sustainable economic development and
growth; the knowledge fabric is what facilitate any long term strategy,
and has been shown in the literature as clear catalyst for societal
prosperity and economic resilience. For this reason the study of
knowledge is detrimental for policy making, and understanding how to
increase its diversity is more relevant than ever before.

Knowledge is a set of information that covers one or many topics, and
its characteristics are contingent on the different forms it can take or
how it was created, generally speaking, academia and businesses are the
main knowledge creators in any economy through research and patents.
Essentially, knowledge can be codified (accessible by anyone through any
medium), or tacit (personal information based on social connections,
intuition, experience, etc, that's hard to share with others). The
consensus in the literature is that the main driver of competitive
advantage for firms is the tacit form of knowledge, which is also widely
acknowledged that it's space dependent. However, the knowledge produced
by firms can be reliably seen in patents, although they capture codified
information, they also reveal tacit knowledge and its geographic
footprint in space. This means that its detrimental to assess the
knowledge in patenting activities (we refer to this knowledge as
technologies), and more so to focus on the local aspect of these
activities i.e: sub-national regions.

This framing, however, is not at all new or novel. In fact, this is the
entire aim of the literature of the geography of innovation; to study
how innovation is created and diffused to different actors in different
geographical contexts. Specifically, relatedness and economic complexity
(REC) is one of the main streams of literature that focus on the
relationships between activities and geographies. The conceptual and
methodological framework that REC provides is widely used and adopted in
academia and among policy practitioners and was one of the main
contributors to the smart specialisation policy literature. The ideas
embedded in this framework, to put it simply, rely on the premise of
spatial dependence of tacit knowledge in local/regional
economies/geographies and focus on simplifying these relationships using
network science to model the relationships between knowledge and
regions. Albeit these simplifications provide valuable scope for
analysis and interpretation, the cost from the loss of granular
information implies that there's much more conceptual, methodological,
and empirical work needed. The reason for this is because the loss of
information bias the empirical interpretation in the sense that we end
up with a homogeneous implication with weak regards to the regional and
national contexts as well as the technological characteristics. This
work is motivated by this gap and aims to simply contextualise the study
of knowledge diversification using the same granular information
publicly available and commonly used in the REC literature. The idea is
simple, account for endogenous and exogenous contexts using granular
data and understand the contexts and contingencies that drives regional
diversification.

\section{Problem statement}\label{problem-statement}

The main problem that this works aims to solve is directly embedded in
the methodological and empirical framework of REC. REC models
diversification through network aggregation based on co-location and
co-occurrence patterns. Using these patterns, different aggregations are
used to quantify relatedness (the frequency of observing a pair of
activities in the same region), and relatedness density (how much of the
activities frequently observed together a region has). However, these
measures are often interpreted not as aggregations of frequent
observations but rather as relationship models. Empirically,
diversification is studied using these constructs as predictors of
entry, that is a region's entry to a new specialisation in a new
technology, often considered as a binary outcome. In here we briefly
outline the big picture in the REC methodological and empirical
framework, its conceptual issues, its empirical consequences, and
highlight the research gap.

\subsection{Relatedness, relatedness density, and
diversification}\label{relatedness-relatedness-density-and-diversification}

Relatedness and relatedness density are essentially measures of
proximity. In a sense they describe how close two technologies are close
to each other, or how close a given technology is to a region given its
portfolio of technologies. To further decompose the problem here, we
will first establish the methodological constructs for proximity
measures. For relatedness that's co-occurrence, and for relatedness
density that's the linear aggregation of relatedness.

First, co-occurrence is essentially the frequency of observing two
activities together. In the REC literature, this frequency describes the
strength of the relationship. Activities frequently observed together
are more related than the pairs rarely observed together.

Second, linear aggregation of relatedness essentially measures the
percentage of co-located technologies in a region that are related to a
reference technology. Thus, we can think about relatedness density as
the link between related technologies and co-located technologies. The
idea in the REC literature states that relative to a given technology,
the more related technologies a region has, the more likely that it can
develop that technology.

These two constructs are used together to predict the probability that a
region will enter a new technology. The REC literature shows that
relatedness density is consistently associated with higher probabilities
in almost all studies. These results among others were one of the major
latent contributors to smart specialisation strategy (S3) policy. Thus
the consensus in the literature was clear: In order to diversify into
new technologies with the highest likelihood of success, regions must
prioritise investment in related technologies.

\subsection{Empirical consequences}\label{empirical-consequences}

The idea of resilience is not a main focus for the REC literature nor it
is ours. However, falling back to this concept allows us to further
assess the empirical consequences of the mainstream interpretation of
relatedness and relatedness density. The idea is that in order to be
resilient to external shocks and subsequent uncertainties
diversification is key. But what kind of diversification is required and
feasible and how to achieve it is the focus here. The REC literature
tells us that the most likely successful diversification strategy is the
one that targets related capacity in the regions, often referred to as
related variety. However, generalising this recommendation is not that
straight forward. Aggregate regional capacities and their national and
broader geographic contexts differ significantly. The initial landscape
of the regional technological portfolio is detrimental here because this
strategy could favour regions with already diverse portfolio but it's
questionable that regions with limited portfolios would equally benefit.
This is aligned with the concept of path dependency, related variety
without context enforces that dependency and locks regions within their
limited capacities. This brings us back to the core focus of in this
work; context is key. However, context on its own here might not be
enough since path dependency of related variety is a direct result of
how relatedness and relatedness density is calculated and interpreted.
The constructs that enable these measures (co-occurrence and linear
aggregation) are the core problem that we're highlighting here. The
reason behind this specific focus relied on the implicit assumptions
embedded in these methodological constructs.

Co-occurrence assumes that technologies frequently observed together are
likely related. Although this is within the boundaries of common sense,
it's highly unlikely that's actually the case. A frequency measure is
only informative when we have more observations than items---that is,
more geographies than technologies. Almost always, we will have more
technologies than geographies, this means that relatedness is at best
noisy. Additionally, relatedness as interpreted in the literature
quantifies the relationship between pairs of technologies. However, as
it stands, there's no differentiation in the direction of that
relationship thus, assuming that the relationship between two
technologies is symmetrical. Albeit this assumption in itself is not
problematic, it exacerbates the linearity issue when we measure
relatedness density as we loose information in co-occurrence, symmetry,
and linear aggregations. This takes us to the final issue we would like
to highlight; relatedness density. Simply put, relatedness density
measures the sum of technologies related to a reference technology
present in a given region. The implicit assumption in here is that
technologies are linked through linear combinations, and those
combinations predict the likelihood of successful diversification.
However, relatedness density is often interpreted as a value that
quantifies the existing requirements a region has relative to a
technology, whereas the sum of existing related technologies do not
inform us on the actual requirements.

In summary, relatedness and relatedness density measures suffer from
diverse methodological issues embedded in the implicit assumptions in
their core constructs. Co-occurrence and linear aggregation of observed
frequency are misinterpreted, accrues information loss, and poorly
handles the granular data often used. This means that the empirical and
methodological work ahead must account for these issues to further
contextualise the study of diversification strategies.

\subsection{Research problem}\label{research-problem}

In the light of all the mentioned in this section, we fall back again on
the core idea that we started this text with; How can we contextualise
diversification strategies? The answer to this question is multi-layered
and complex. In this section we started by outlining the importance of
diversification strategies for regions into new technologies via
patenting activities. We explained that the REC literature provides
interesting methodological and conceptual framework of analysis and
showed that despite their usefulness they suffer from structural issues
that limit the advantages of the used granular data, thus limit the
incorporation of a broader context empirically. Essentially, the
research problem we focus on here, is both methodological and empirical
in nature. We highlighted the structural methodological issues as the
research gap which will be the focus of our methodological and empirical
contribution.

\includegraphics[width=9.3in,height=17.96in]{intro_files/figure-latex/mermaid-figure-1.png}

\section{Objective}\label{objective}

The objective from this work is to extend the methodological framework
of relatedness starting from the structural issues it suffers from. In
order to show how our methodological contribution benefit the study of
regional technological diversification we rely on an empirical study
that shows how broader context can be included and how such context
inform granular policy insights. The broader context we aim to include
empirically relied on relaxing the implicit relatedness assumptions,
explicitly include the endogenous characteristics of the technologies
and regions contingent on the regional knowledge infrastructure while
simultaneously account for the characteristics of the national
ecosystems. With such an approach we end up accounting for more
contextual layers than the mainstream approach.

\section{Research questions and
hypothesis}\label{research-questions-and-hypothesis}

\textbf{RQ1: How to further contextualise diversification strategy based
on the relatedness framework?}

\begin{itemize}
\tightlist
\item
  H1a: Asymmetric relatedness measures better predict technology entry
  than symmetric measures.
\end{itemize}

\textbf{RQ2: Is successful diversification contingent on regional
knowledge infrastructure?}

\begin{itemize}
\item
  H2a: Technology-specific characteristics impact on diversification is
  contingent on the regional knowledge coherence.
\item
  H2b: Technology-specific characteristics impact on diversification is
  contingent on the regional knowledge stock.
\end{itemize}

\textbf{RQ3: Does the national ecosystem influence diversification?}

\begin{itemize}
\tightlist
\item
  H3a: National Entrepreneurial Ecosystem characteristics positively
  affects regional entry into new technologies.
\end{itemize}

\textbf{RQ4: What role does space have? Do neighbouring regions
influence diversification?}

\begin{itemize}
\tightlist
\item
  H4a: Spatial spillovers of outcome from neighbouring regions
  positively affect technology entry.
\item
  H4b: Neighbouring regions' technological characteristics influence
  focal region diversification outcomes.
\item
  H4c: Geographic proximity to regions with coherent knowledge
  infrastructure increases entry probability.
\end{itemize}

\section{Structure}\label{structure}

We structure this work around our methodological and empirical
contributions. Given the complex nature of the problems we aim to solve,
we will first start with more literature context that underlines
geography of innovation, and more importantly relatedness and economic
complexity literatures. We then outline our approach in the methodology
chapter where we discuss in more details measures of relatedness and
proximity and provide alternative conceptualisation to modeling the
relationships between technologies and regions. In the same chapter we
also detail other elements that will be relevant to the empirical part
specifically knowledge coherence. In the empirical chapter we outline
the research design and the modeling procedures that implements our
methodological contribution. The results chapter outline the results of
our work and its empirical consequences and interpretation, which we
discuss in more detail in the discussion and conclusion of this work in
which we also outline different further extensions and future research
directions.

\section{Relatedness(?)}\label{relatedness}

\bookmarksetup{startatroot}

\chapter{Literature review}\label{literature-review}

\begin{figure}[H]

{\centering \pandocbounded{\includegraphics[keepaspectratio]{pres_rf_s7/get-out-of-my-head-meme.gif}}

}

\caption{\textbf{THIS IS HARD!}}

\end{figure}%

\bookmarksetup{startatroot}

\chapter{Relatedness in the era of Machine
Learning}\label{relatedness-in-the-era-of-machine-learning}

When two technologies (or products, activities, etc) share similar sets
of input requirements (knowledge, resources, etc), we qualify them as
related (Hausmann and Hidalgo 2011). This empirical observation has been
confirmed in many areas in different streams of literature and
formalised in the Principle of Relatedness(PoR) (Hidalgo et al. 2018).
However, deriving concrete policy implications from this principle has
proven far from straightforward (Hidalgo 2023; Li and Neffke 2024). The
PoR is essentially a framework that formalises a qualitative intuition
that's been present in various streams of literature: The industrial
fabric in a geographic location matters (Hidalgo 2021; Hidalgo and
Hausmann 2009). This framework enables researchers to derive different
metrics that quantify path dependency and, therefore, infer more
granular and pragmatic policy recommendations (Li and Neffke 2024).
However, we often find in the literature that many studies refrain from
investigating beyond the identification of path dependencies(Hidalgo
2023). Although such identification may prove interesting at times, the
entire idea of the PoR is to be used to break free from the path
dependency curse and focus on unique regional paths that promote
diversification (Imbs and Wacziarg 2003). Diversification is the
endpoint because it creates different sets of non-fungible tacit and
non-tacit knowledge/capacity (Collins 1974) that can be compounded over
time and across industries to create value that drives regional growth
and development (Dosi 1982; Weitzman 1998). Knowledge, in all its forms,
is the driver of the PoR policy implications (P.-A. Balland and Boschma
2022; P.-A. Balland et al. 2019). And although identifying promising
areas of knowledge(related or unrelated) is a useful exercise. Figuring
out the unique elements that dictate the dynamics of knowledge flows is
at the heart of the industrial policies in this context (Nomaler and
Verspagen 2024). Moreover, the PoR is also complemented by the Economic
Complexity paradigm(EC) pioneered for the first time in (Hidalgo and
Hausmann 2009). EC is a methodological framework that builds on the PoR
and frames economies as complex systems. The idea is simple: an economy,
regardless of its scale, is a complex system that might be impossible to
determine the entirety of its components. But if we quantify the
interactions between different systems and their different components,
then we can estimate indices and metrics that capture most of the
variation. In this sense, the PoR quantifies path dependency
patterns(via metrics like relatedness and proximity), and EC quantifies
the sophistication of specialisation patterns(via metrics like
complexity and fitness)\footnote{We will use the terms relatedness and
  complexity to refer to these two dimensions moving forward following
  the literature nomenclature}. One can also describe relatedness as a
variation of a recommendation system and complexity as a dimensionality
reduction exercise. However, there's still no general consensus on the
reliability of any given methodology for both exercises, regardless of
the popularity of one or the other. What there's a consensus on,
however, is that these frameworks and the toolbox they provide can be
improved further as stated in C. Pinheiro (2025). Regardless, EC and PoR
were adopted in many policy papers such as Zaccaria et al. (2018), E and
A (2021) and G, D, and L (2025).

In this context, the literature provides more threads of ideas that
target a deeper understanding and analysis of complex economic systems.
For instance, investigations and studies regarding unrelated
diversification (Flávio L. Pinheiro et al. 2022; Boschma et al. 2023),
geographic inequalities (Flavio L. Pinheiro et al. 2025; Hartmann et al.
2017), emerging industries/technologies (C. Lee et al. 2018; Fessina et
al. 2024), and diversification strategies (Alshamsi, Pinheiro, and
Hidalgo 2018), among others, are pioneering the effort to bridge
different gaps in theory, policy implications, and methodology. These
ideas, among others, suggest that investing in unrelated activities can
yield greater value and help break free from the path‐dependency
curse---a phenomenon the literature shows exacerbates regional
inequalities. Thus, one of the challenges is to quantify how to expand
related activities beyond path dependency(strategy), which
activity/sector to aim for(target), and the requirements for such an
investment to be fruitful(condition). The scope of our study is
expanding ideas around diversification strategies and conditions.

Policy makers often face a difficult choice when deciding on industrial
upgrading. The first is to take advantage of existing local capacity and
knowledge(related diversification). This choice is presumably the
easiest one, since the local economy already has what it takes for the
implementation (Boschma 2017). The second is to invest in building new
capacity/knowledge(unrelated diversification) with all the risks that
such a gamble accommodates (Coniglio et al. 2021). These choices have
been at the centre of different theories in development economics(big
push, forward/backward linkages (Rosenstein-Rodan 1943; Hirschman 1958),
etc). However, we argue on the side of (C. Pinheiro 2025) that the basis
of this narrative is incomplete since it contains an implicit assumption
that is often overlooked: it's easier for an economic system to
diversify into a related activity than an unrelated one. But how do we
assess the ease of diversification, accounting for its level of
relatedness?

In this paper, we propose a measure that quantifies the ease of
diversification of an economic system agnostic to the level of
relatedness of an activity. We believe that our approach accommodates
this assumption explicitly and capture all possible diversification
paths be it related or unrelated. Additionally, we extend our
contribution by assessing how different classical socio-economic factors
influence the ease of diversification. The remainder of the paper is
organised as follows: we first introduce the data, then we present the
methodology for each phase of our work. Then we present our results, and
we conclude this work with a discussion.

K. Lee and Malerba (2017) show that for every stage of industrial
maturity different level of diversity among other ``initial-conditions''
is needed. The main idea is that local capacities are necessary but
insufficient condition for related diversification. This has been
formalised explicitly in (Hausmann and Hidalgo 2011). where the authors
define the ``quiescence trap'', which can be observed when a country
with few capabilities face low incentive to accumulate new ones.
Additionally, in a broader policy context, and even if we assume that
these initial conditions are met locally, and related diversification is
feasible, it may simply exacerbate the regional disadvantages. Indeed
related diversification has been observed to increase the gaps between
locations (Mealy and Coyle 2022; Flavio L. Pinheiro et al. 2025).
Essentially, if we have two locations, one already has a range of
complex capacities, the other doesn't. If policy only backs what each
location already does well, the complex location keeps getting ever more
complex, drawing more investment and talent, while the other one falls
further behind exacerbating inequality. Moreover, there's no consensus
on one way to quantify relatedness. The literature usually relies on the
co-occurrence matrix to construct the relatedness between
products/technologies etc. (Coniglio et al. 2021) point out that most
studies in this context do not differentiate between random
co-occurrences, and co-occurrences that are due to related capacities
and proposes a test to investigate these significance of these
relationships. Similarly, the proposed methodology in (Albora et al.
2023) responds to the same criticism and argue that the number of
products/technologies almost always outnumber that of regions/locations,
therefore the information extracted from the co-occurrence matrix is at
best a random walk.

We use Data from the European Patent Office, which contains details on
patent applications from 1978 to 2021. The EPO provides a rigorous and
detailed classification of each patent application up to 8 or more
digits. In our case, we consider the IPC classifications, but since
these classifications are extremely granular and are considerably larger
than the regions observed (at 8000+ classes), we limit our data to the
4th digit of the IPC classification. These 4 digits contain 3 layers of
information with which we can define a given technology, a section
denoted by a letter, a class denoted by two digits, and a subclass
denoted by another letter. Thus, an IPC class/technology such as F16H is
structured hierarchically: Section F covers Mechanical engineering
(including lighting, heating, weapons, and blasting), Class 16 pertains
to engineering elements and general methods for producing and
transmitting mechanical power, and Subclass H specifically addresses
gears, shaft connections, and gearing for conveying rotary motion. With
such a subset, we ended up with 641 distinct technologies. Additionally,
the same data also provides details on where the applications were made,
we capture these details at the NUTS2 level\footnote{With the exception
  of Belgium and the United Kingdom who were included at NUTS1 level}
for 34 European countries within and outside the European Union,
spanning across 345 regions. Additionally we also use data from the
Eurostat database to incorporate regional level socio-economic factors
which are detailed further in section \textbf{?@sec-factors} and
summarised in \textbf{?@tbl-sum}.

Furthermore, we quantify regions' specialisation by means of the
Revealed Comparative Advantage(RCA) (Balassa 1965). In our context, the
RCA measures the region's relative specialization level in a given
technology, which enables us to capture both expertise and diversity
when we aggregate all the technologies for each region. This measure,
also known as the Balassa index, proved useful in determining complex
and non-linear relationships between products/activities. Although the
RCA is mainly designed for use with international trade data, it has
also been adopted in the literature on the geography of innovation.
Simply put, the RCA quantifies simultaneously the relative level and the
quality of co-occurrence, which reduces the noise in the data. Although
some papers criticise the use of the RCA with patent classes (P. Balland
and Boschma 2019; Diodato et al. 2023), we think it fits our objective
in capturing meaningful relationships between technologies. We compute
these RCA measures to obtain, for each year, a matrix denoting the
regions in its rows and the technologies in its columns. We formalize it
as follows: Let \(X_{r,t,y}\) be the measure of activity (patent counts)
of region \(r\) in technology \(t\) during year \(y\). Where
\(\mathcal{T}\) is the set of technologies, \(\mathcal{Y}\) is the set
of years, \(\mathcal{R}\) is the set of regions, and \(\mathcal{C}\) is
the set of countries, such that:

\[
\mathcal{T} = \{\,t : 1 \le t \le N_T\},\quad
\mathcal{Y} = \{\,y : 1 \le y \le N_Y\},\quad
\mathcal{R} = \{\,r : 1 \le r \le N_R\},\quad
\mathcal{C} = \{\,c : 1 \le c \le N_C\}.
\]

And \(N_T, N_Y, N_R, N_C\) are the total counts of technologies, years,
regions and countries.

The RCA of region \(r\) in technology \(t\) in year \(y\) is

\[
\mathrm{RCA}_{r,t,y}
= \frac{\displaystyle\frac{X_{r,t,y}}{\sum_{t'} X_{r,t',y}}}
       {\displaystyle\frac{\sum_{r'} X_{r',t,y}}{\sum_{r',t'} X_{r',t',y}}}
= \frac{X_{r,t,y}\,\sum_{r',t'}X_{r',t',y}}
       {\bigl(\sum_{t'}X_{r,t',y}\bigr)\,\bigl(\sum_{r'}X_{r',t,y}\bigr)}
\]

For each year \(y\), we then assemble the \textbf{RCA matrix}
\(\mathbf{R}^{(y)}\) whose \((r,t)\)-entry is \(\mathrm{RCA}_{r,t,y}\):

\[
\mathbf{R}^{(y)}
= \bigl[\mathrm{RCA}_{r,t,y}\bigr]_{r=1,\dots,N_{R}}^{t=1,\dots,N_{T}}
\]

These yearly measures are essential for us, since our entire approach
depends on different manipulations around these stacked matrices.

\bookmarksetup{startatroot}

\chapter{Technological potential}\label{technological-potential}

\section{Methodology}\label{methodology}

Our multi-phase approach builds on different methods in the literature.
In the first phase we model the relationship between the
technologies\footnote{based on their measures of RCA} using multiple
random forest algorithm models ((Breiman 2001)). From these models we
obtain probabilities that denote the chance that a given region will
develop expertise in a given technology conditional on its past capacity
in all other technologies\footnote{The different phases of this study
  rely on smaller portions of this data. In the first stage we take the
  patent data as is, in the second stage remove the outliers and the
  regions/countries with more than 2 years of missing data points, then
  for the regression we had to shrink the data further because of the
  unavailability of the data for 4 years and for many regions despite
  our attempts to impute the missing values with a regression tree}.
With this we have two elements: the first is the actual current observed
capacity of a region(\(\mathbf{R}^{(y)}\)); the second is the current
potential that a region will develop expertise based on its past
technological capacity. We call that measure regional technological
potential. We leverage these two elements in constructing the second
phase. Essentially we estimate the distance between the (hypothetical)
potential of region and its actual observed capacity in developing
expertise in a given technology. After countless iterations, we decided
that a Stochastic Frontier Analysis (hereafter SFA) approach is adequate
for this phase ((Afriat 1972; Aigner, Lovell, and Schmidt 1977)). This
distance is an in/efficiency estimate that quantifies what we call the
regional friction. We define regional friction as the effort needed for
a region to leverage it's past technical capacities to develop new
current capacities. This means that the most efficient regions are the
closest to their estimated potential. When a region's observed expertise
is close to its potential it means that its capable to absorb knowledge
and internalise expertise in its economic fabric. In a final stage we
regress the estimates of regional frictions via OLS to model and test
the influence of diverse socio-economic factors that relate mainly to
knowledge, infra-structure, among other economic factors that we will
elaborate on in the coming section.

\subsection{Regional technological
potential}\label{regional-technological-potential}

We follow the methodology proposed by (Albora et al. 2023) for trade
data. The methodology named product progression, is based on a machine
learning approach that enables researchers to unravel novel aspects of
their RCA data. In our case that would be the non-linear dependence
between technologies. Our Objective from this phase is to eventually
predict whether a region will develop an expertise in a given
technology. For reasons of data availability in other data bases that we
will use in the next stages, we opted to limit these predictions to an
11 years interval spanning from 2008 to 2018 using data from 4 years ago
for each prediction. These predicted probabilities are precisely the
regional technological potential. They denote a hypothetical situation
that describe for each region, the technologies it has potential to
develop expertise in given the relationships that we already modeled.

The modeling of the Random Forest algorithm is not intuitive in this
proposed methodology. In fact the novelty of the approach proposed by
(Albora et al. 2023) is not the use of a tree-based algorithm, but
rather to model each technology separately. The idea is to construct a
model for every technology in \(\mathbf{R}^{(y)}\) matrix such that the
target technology \(i, i\in \mathcal{T}\) is the outcome and the
features are all other technologies different than \(i\). The trick here
is to binarise the outcome and leave the features as they are for every
model we train such that:

\[
z_{r,t,y}
\;=\;
\begin{cases}
1 & \text{if }\mathrm{RCA}_{r,t,y}\ge1 \\
0 & \text{otherwise}
\end{cases}
\]

The \(z_{r,t,y}\) term reflect the capacity of region \(r\) at year
\(y\) for technology \(t\). In here capacity means that a region has an
advantage/specialised in that specific technology relative to the other
regions. Additionally we include a 4 year lag, or a fixed horizon we
call delta, \(\delta=4\) in the features since the entire idea is to
assess the capacity of the current outcomes based on the past features.
This aligns with instincts in the literature in which studies like
(Andreoni and Chang 2019) posit that past capacities predict future
diversification. Eventually the predicted outcomes describe what
technology is possible to develop expertise in, given the observed
capacity \(\delta\) years ago. However, as stated in Albora et al.
(2023) choosing the value of \(\delta\) is challenging since increasing
it decrease the performance of the models. Our choice here, relies on
this observation and is the most optimal decision since we train the
models for different years instead of just one, thus we need to have for
each model at each year of prediction enough observations.

The training and testing sets are constructed consequently:

We use a fixed horizon (\(\delta=4\)) years to predict future expertise.
Let years run from \(y = y_0...y_f\), where \(y_0 = 1978\) and
\(y_f = 2018\), let's also consider the target year of prediction
\(y_t \in \{2008,..., 2018\}\). We then have:

\[
X_{\text{train}} = \{ \mathrm{RCA}_{r,t,y} | y \in [y_0, y_t - 2\delta]\},\quad Y_{\text{train}} = {z}_{r,t',y} | y \in [y_0 + \delta,  y_t - \delta] \}
\]

\[
X_{\text{test}} =  \{ \mathrm{RCA}_{r,t,y} |y_t - \delta\},\quad 
Y_{\text{test}} = \{ z_{r,t,y} |y_t\}
\] Given the complexity of computations, which would require infeasible
timing, we conducted cross validation on a random sample of models
targeting G06G (Analog computers for data processing), B67B (Closing
bottles, jars, or similar containers), D02J (Mechanical finishing or
refining of yarns), and C08J (Working-up plastics-processing, recovery,
or treatment of waste). Then we chose the parameter values with the most
frequency and used them for the rest of the models specifically: mtry =
139, trees = 100, min\_n = 38. For our case and computational
constraints, this was the only feasible approach. The training was
conducted in R using the Ranger package (Wright and Ziegler 2017), with
targets (Landau 2021) as a pipeline orchestrator and the tidymodels
framework.

Once we train our models one for each technology for each of the 11
years (7051 models in total) we obtain at each year, and for each
technology, the probability that a region develops expertise. We define
this set of probabilities as \(P(\mathrm{RCA}_{r,t,y} \geq 1)\) which we
write simply as \(p_{r,t,y}\) and we will refer to these probabilities
as the regional technological potential. When we aggregate these
probabilities regionally, we obtain the (average) regional potential
\(p_{r,y} = \frac{\sum_{t} p_{r,y,t}}{n_t}\), with \(n_t\) the
corresponding number of observations.

\bookmarksetup{startatroot}

\chapter{Technology space}\label{technology-space}

\subsection{Feature Importance Technolgy
Space}\label{feature-importance-technolgy-space}

A natural approach to constructing technology networks would be to use
patent citation data: if Patent A cites Patent B, and they contain
different technologies, this reveals a relationship between those
technology domains. However, this approach presents three critical
limitations for our purposes. First, EPO citations include
examiner-added citations that may not reflect actual knowledge flows
between inventors. Second, aggregating patent-level citations to
technology-level relationships relies on co-occurrence patterns that
obscure the directionality of knowledge dependencies---whether
technology A enables B or vice versa. Third, citation networks are
inherently backward-looking, reflecting past relationships rather than
predicting future technological trajectories.

To address these limitations, we adopt the Feature Importance Product
Space (FIPS) methodology (Fessina et al. 2024). Rather than inferring
relationships from citation co-occurrence, FIPS uses machine learning to
identify which existing technology specializations predict future
specialization in other technologies. Specifically, we train random
forest models where the current presence of regional expertise in
technology T is predicted by past expertise patterns across all other
technologies. The resulting feature importance scores reveal
directional, predictive relationships: if expertise in technology A
strongly predicts future expertise in B, this indicates A is a
``stepping stone'' toward B, even if the reverse is not true. This
asymmetric structure captures hierarchical technological dependencies
that symmetric co-occurrence measures would miss, and its
forward-looking nature aligns with our focus on how current network
positions influence future patent value.

\pandocbounded{\includegraphics[keepaspectratio]{resources/technology_network.png}}

Following (Fessina et al. 2024), we quantify specialization patterns by
means of the Revealed Comparative Advantage(RCA) (Balassa 1965).
Although the RCA is mainly designed for use with international trade
data, it has also been adopted in the literature of the geography of
innovation following the Balland nomenclature
(\textbf{balland2017geography?}) we will refer to this metric as the
Revealed Technological Advantage(RTA) instead. The RTA measures a
location's relative specialization level in a given technology which
enables us to capture both expertise and diversity when we aggregate all
the technologies for each location. This measure, proved useful in
determining complex and non-linear relationships between
products/activities. Simply put, the RTA quantifies simultaneously the
relative level and the quality of co-occurrence, which reduces the noise
in the network data. Although some papers criticise the use of the
RCA/RTA with patents classes {[}pinheiro2025{]}, we think it fits our
objective in capturing meaningful relationships between technologies
when we use them in the framework of (Fessina et al. 2024). We compute
these measures to obtain for each year a matrix denoting the regions in
its rows and the technologies in its columns. We formalize it as
follows: Let \(X_{r,t,y}\) be the measure of activity (patent counts) of
region \(r\) in technology \(t\) during year \(y\). Where
\(\mathcal{T}\) is the set of technologies, \(\mathcal{Y}\) is the set
of years, and \(\mathcal{R}\) is the set of regions.

\bookmarksetup{startatroot}

\chapter{Diversification, coherence, and the role of
space}\label{diversification-coherence-and-the-role-of-space}

\bookmarksetup{startatroot}

\chapter{Summary}\label{summary}

In summary, this book has no content whatsoever.

\bookmarksetup{startatroot}

\chapter*{References}\label{references}
\addcontentsline{toc}{chapter}{References}

\markboth{References}{References}

\phantomsection\label{refs}
\begin{CSLReferences}{1}{0}
\bibitem[\citeproctext]{ref-afriat1972}
Afriat, Sidney N. 1972. {``Efficiency Estimation of Production
Functions.''} \emph{International Economic Review}, 568--98.

\bibitem[\citeproctext]{ref-aigner1977}
Aigner, Dennis, CA Knox Lovell, and Peter Schmidt. 1977. {``Formulation
and Estimation of Stochastic Frontier Production Function Models.''}
\emph{Journal of Econometrics} 6 (1): 21--37.

\bibitem[\citeproctext]{ref-albora2023product}
Albora, Giambattista, Luciano Pietronero, Andrea Tacchella, and Andrea
Zaccaria. 2023. {``Product Progression: A Machine Learning Approach to
Forecasting Industrial Upgrading.''} \emph{Scientific Reports} 13 (1):
1481.

\bibitem[\citeproctext]{ref-alshamsi2018optimal}
Alshamsi, Aamena, Flávio L Pinheiro, and Cesar A Hidalgo. 2018.
{``Optimal Diversification Strategies in the Networks of Related
Products and of Related Research Areas.''} \emph{Nature Communications}
9 (1): 1328.

\bibitem[\citeproctext]{ref-andreoni2019political}
Andreoni, Antonio, and Ha-Joon Chang. 2019. {``The Political Economy of
Industrial Policy: Structural Interdependencies, Policy Alignment and
Conflict Management.''} \emph{Structural Change and Economic Dynamics}
48: 136--50.

\bibitem[\citeproctext]{ref-balassa1965rca}
Balassa, Bela. 1965. {``Trade Liberalisation and {`Revealed'}
Comparative Advantage 1.''} \emph{The Manchester School} 33 (2):
99--123.

\bibitem[\citeproctext]{ref-balland2019beyond}
Balland, PA, and R Boschma. 2019. {``Smart Specialization: Beyond
Patents.''} \emph{European Commission: Brussels, Belgium}.

\bibitem[\citeproctext]{ref-balland2022scientific}
Balland, Pierre-Alexandre, and Ron Boschma. 2022. {``Do Scientific
Capabilities in Specific Domains Matter for Technological
Diversification in European Regions?''} \emph{Research Policy} 51 (10):
104594.

\bibitem[\citeproctext]{ref-balland2019smart}
Balland, Pierre-Alexandre, Ron Boschma, Joan Crespo, and David L Rigby.
2019. {``Smart Specialization Policy in the European Union: Relatedness,
Knowledge Complexity and Regional Diversification.''} \emph{Regional
Studies} 53 (9): 1252--68.

\bibitem[\citeproctext]{ref-boschma2017relatedness}
Boschma, Ron. 2017. {``Relatedness as Driver of Regional
Diversification: A Research Agenda.''} \emph{Regional Studies} 51 (3):
351--64.

\bibitem[\citeproctext]{ref-boschma2023role}
Boschma, Ron, Ernest Miguelez, Rosina Moreno, and Diego B
Ocampo-Corrales. 2023. {``The Role of Relatedness and Unrelatedness for
the Geography of Technological Breakthroughs in Europe.''}
\emph{Economic Geography} 99 (2): 117--39.

\bibitem[\citeproctext]{ref-RF2001}
Breiman, Leo. 2001. {``Random Forests.''} \emph{Machine Learning} 45:
5--32.

\bibitem[\citeproctext]{ref-collins1974tea}
Collins, Harry M. 1974. {``The TEA Set: Tacit Knowledge and Scientific
Networks.''} \emph{Science Studies} 4 (2): 165--85.

\bibitem[\citeproctext]{ref-coniglio2021evolution}
Coniglio, Nicola D, Davide Vurchio, Nicola Cantore, and Michele Clara.
2021. {``On the Evolution of Comparative Advantage: Path-Dependent
Versus Path-Defying Changes.''} \emph{Journal of International
Economics} 133: 103522.

\bibitem[\citeproctext]{ref-diodato2023economic}
Diodato, Dario, Lorenzo Napolitano, Emanuele Pugliese, and Andrea
Tacchella. 2023. {``Economic Complexity for Regional Industrial
Strategies.''} Joint Research Centre.

\bibitem[\citeproctext]{ref-dosi1982technological}
Dosi, Giovanni. 1982. {``Technological Paradigms and Technological
Trajectories: A Suggested Interpretation of the Determinants and
Directions of Technical Change.''} \emph{Research Policy} 11 (3):
147--62.

\bibitem[\citeproctext]{ref-albora2021economic}
E, Pugliese, and Tacchella A. 2021. {``Economic Complexity Analytics:
Country Factsheets.''} \url{https://doi.org/10.2760/368138}.

\bibitem[\citeproctext]{ref-fessina2024identifying}
Fessina, Massimiliano, Giambattista Albora, Andrea Tacchella, and Andrea
Zaccaria. 2024. {``Identifying Key Products to Trigger New Exports: An
Explainable Machine Learning Approach.''} \emph{Journal of Physics:
Complexity} 5 (2): 025003.

\bibitem[\citeproctext]{ref-albora2025economic}
G, Albora, Diodato D, and Napolitano L. 2025. {``Economic Complexity
Analytics: Country Factsheets 2024.''}
\url{https://doi.org/10.2760/9201933}.

\bibitem[\citeproctext]{ref-hartmann2017linking}
Hartmann, Dominik, Miguel R Guevara, Cristian Jara-Figueroa, Manuel
Aristarán, and César A Hidalgo. 2017. {``Linking Economic Complexity,
Institutions, and Income Inequality.''} \emph{World Development} 93:
75--93.

\bibitem[\citeproctext]{ref-hausmann2011network}
Hausmann, Ricardo, and César A Hidalgo. 2011. {``The Network Structure
of Economic Output.''} \emph{Journal of Economic Growth} 16: 309--42.

\bibitem[\citeproctext]{ref-hidalgo2021economic}
Hidalgo, César A. 2021. {``Economic Complexity Theory and
Applications.''} \emph{Nature Reviews Physics} 3 (2): 92--113.

\bibitem[\citeproctext]{ref-hidalgo2023policy}
---------. 2023. {``The Policy Implications of Economic Complexity.''}
\emph{Research Policy} 52 (9): 104863.

\bibitem[\citeproctext]{ref-hidalgo2018principle}
Hidalgo, César A, Pierre-Alexandre Balland, Ron Boschma, Mercedes
Delgado, Maryann Feldman, Koen Frenken, Edward Glaeser, et al. 2018.
{``The Principle of Relatedness.''} In \emph{Unifying Themes in Complex
Systems IX: Proceedings of the Ninth International Conference on Complex
Systems 9}, 451--57. Springer.

\bibitem[\citeproctext]{ref-hidalgo2009building}
Hidalgo, César A, and Ricardo Hausmann. 2009. {``The Building Blocks of
Economic Complexity.''} \emph{Proceedings of the National Academy of
Sciences} 106 (26): 10570--75.

\bibitem[\citeproctext]{ref-hirschman1958strategy}
Hirschman, Albert O. 1958. {``The Strategy of Economic Development.''}

\bibitem[\citeproctext]{ref-imbs2003stages}
Imbs, Jean, and Romain Wacziarg. 2003. {``Stages of Diversification.''}
\emph{American Economic Review} 93 (1): 63--86.

\bibitem[\citeproctext]{ref-landau2021targets}
Landau, William Michael. 2021. {``The Targets r Package: A Dynamic
Make-Like Function-Oriented Pipeline Toolkit for Reproducibility and
High-Performance Computing.''} \emph{Journal of Open Source Software} 6
(57): 2959.

\bibitem[\citeproctext]{ref-lee2018early}
Lee, Changyong, Ohjin Kwon, Myeongjung Kim, and Daeil Kwon. 2018.
{``Early Identification of Emerging Technologies: A Machine Learning
Approach Using Multiple Patent Indicators.''} \emph{Technological
Forecasting and Social Change} 127: 291--303.

\bibitem[\citeproctext]{ref-lee2017catch}
Lee, Keun, and Franco Malerba. 2017. {``Catch-up Cycles and Changes in
Industrial Leadership: Windows of Opportunity and Responses of Firms and
Countries in the Evolution of Sectoral Systems.''} \emph{Research
Policy} 46 (2): 338--51.

\bibitem[\citeproctext]{ref-li2024evaluating}
Li, Yang, and Frank MH Neffke. 2024. {``Evaluating the Principle of
Relatedness: Estimation, Drivers and Implications for Policy.''}
\emph{Research Policy} 53 (3): 104952.

\bibitem[\citeproctext]{ref-mealy2022them}
Mealy, Penny, and Diane Coyle. 2022. {``To Them That Hath: Economic
Complexity and Local Industrial Strategy in the UK.''}
\emph{International Tax and Public Finance} 29 (2): 358--77.

\bibitem[\citeproctext]{ref-nomaler2024related}
Nomaler, Önder, and Bart Verspagen. 2024. {``Related or Unrelated
Diversification: What Is Smart Specialization?''} \emph{Structural
Change and Economic Dynamics} 70: 503--15.

\bibitem[\citeproctext]{ref-pinheiro2025}
Pinheiro, Cristina. 2025. {``Relatedness and Economic Complexity as
Tools for Industrial Policy: Insights and Limitations.''}
\emph{Structural Change and Economic Dynamics} 72: 1--10.

\bibitem[\citeproctext]{ref-pinheiro2025dark}
Pinheiro, Flavio L, Pierre-Alexandre Balland, Ron Boschma, and Dominik
Hartmann. 2025. {``The Dark Side of the Geography of Innovation:
Relatedness, Complexity and Regional Inequality in Europe.''}
\emph{Regional Studies} 59 (1): 2106362.

\bibitem[\citeproctext]{ref-pinheiro2022time}
Pinheiro, Flávio L, Dominik Hartmann, Ron Boschma, and César A Hidalgo.
2022. {``The Time and Frequency of Unrelated Diversification.''}
\emph{Research Policy} 51 (8): 104323.

\bibitem[\citeproctext]{ref-rosenstein1943problems}
Rosenstein-Rodan, Paul N. 1943. {``Problems of Industrialisation of
Eastern and South-Eastern Europe.''} \emph{The Economic Journal} 53
(210-211): 202--11.

\bibitem[\citeproctext]{ref-weitzman1998recombinant}
Weitzman, Martin L. 1998. {``Recombinant Growth.''} \emph{The Quarterly
Journal of Economics} 113 (2): 331--60.

\bibitem[\citeproctext]{ref-wright2017ranger}
Wright, Marvin N, and Andreas Ziegler. 2017. {``Ranger: A Fast
Implementation of Random Forests for High Dimensional Data in c++ and
r.''} \emph{Journal of Statistical Software} 77: 1--17.

\bibitem[\citeproctext]{ref-zaccaria2018integrating}
Zaccaria, Andrea, Saurabh Mishra, Masud Z Cader, and Luciano Pietronero.
2018. {``Integrating Services in the Economic Fitness Approach.''}
\emph{World Bank Policy Research Working Paper}, no. 8485.

\end{CSLReferences}




\end{document}
